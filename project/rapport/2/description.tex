%Description - Contains A description of how to make to script files battle each other, 

\section{ Description of WAR }
WAR is a basic scripting language, enabling scripters to script out battles, and let them carry out on their own without them. The winner of the simulation will be the one with the best script. \\
	\subsection{The simulation}
		To run a simulation in the WAR simulator a configuration file and at least two team files has to be fed to the simulation engine. 
		These files are written in the WAR scripting language.
		
		The basic rules of the game are defined in the configuration file. 
		This configuration file may be worked out in cooperation by two participants, each representing their own team, 
		to establish ground rules for the war simulation. Such configurations should be, but not limited to, 
		the size of the grid the simulation is going to take place on, how many units each team may own, and when or how the game ends.
		The thought of this is to allow for some flexibility in altering the simulation from game to game.
		Each participant will make a script file they will give to the simulator, 
		which then will interpret the script and simulate the battle. 
		These script files contain how many units and what kind of units the participant would like to use. 
		They may also contain information regarding the behaviour of specific regiments. 
		Too ensue that the battle between two scripts will be fair certain limits on the regiments are specified in the config file.
	
		The one who made the best script will win the game, so every participant has to make the best script, 
		and place the units in the best location on the grid.
	
		When both players are coding their scripts, they will provide the script-files with information about their regiments, 
		but if the player chooses to leave the information out, or even forgets to specify it, 
		the simulator will assign a default value. 
		This helps both rookie scripters and forgetful scripters to provide the simulator 
		with the amount of information required to initiate the simulation.

	\subsection{The language}
		The WAR language will be a simple imperative scripting language. 
		We want the language to be easy to program, such that inexperienced programmers will be able to make a simulation, which is why we chose to
		make the language very simple. The imperative paradigm is what we felt best supported the idea of a simple programming language. We don't need the
		powerful abstractions of object orientated programming nor the task-solving capabilities of the functional languages.
		
		 
	\subsection{Definition of the terms in WAR}
	In this section we will define the different terms of the simulation and of the terms seen in the config- and team file.
	
		\subsubsection{Terms of the simulation}
		
		\subsubsection{Team}		
		A team consists of 1 or more regiments. In the simulation two or more teams will battle to win the simulation.
		
		\subsubsection{Turn}
		Each of the regiment will have a turn each round where they can perform a Actions. The turn will end if the regiment is out of
		action or if the regiment doesn't want to perform any Actions.
		
		\subsubsection{Round}
		A simulation consists of 1 or more rounds. When a round begins each regiment will be given a turn. 
		A new round will start when every regiment have used their turns.

		\subsubsection{Behaviour}
		Behaviour allows the scripter to define how each regiments behaves. 
		By using control statements the scripter may make the regiment behave in a defensive or offensive behaviour, 
		using the provided methods in the language. As an example you may instruct a ranged regiment to behave in such a way, 
		as to not approach another regiment, and only open fire if that regiment is within the range of your archer regiment.
				
		\subsubsection{Grid}
		The grid is a 2 dimensional map where the teams will battle.		
		
%		The grid, might be seen as a map, table, or battlefield where the 'battle' will take place. 
%		Each 'Grid' will have a name, and certain properties which \textbf{must} be defined in the configuration file.
																		
		\subsubsection{Terms in the team file }
	
		\subsubsection{Size}
		The size of the regiment is, in abstraction, the number of soldiers in the current regiment. 
		One might consider it as the overall health of the regiment. While size may be an abstraction of health, 
		it's important to notice that a regiment with a larger size takes up more integers on the grid.
		
		\subsubsection{Type}
		The Type defines how a regiment attacks. A regiments Type can only be Melee or Ranged.
		
		\subsubsection{Range}
		Range defines how far a regiment of Type Ranged can attack. 
		If a Range is defined for a regiment of type Melee, this is simply ignored.
		
		\subsubsection{Damage}
		The damage of a regiment determines how many units in the opposing regiment 
		will be lost per attack performed in the current round of the simulation.
		
		\subsubsection{Movement}
		Movement defines how many tiles a regiment is allowed to move.
		
		\subsubsection{Attackspeed}
		Attackspeed governs how many attacks are performed by each round, giving units further interesting properties. 
		However, in the context of the simulation, the attackspeed is quite simply a multiplier of the damage of Attack.
		
		\subsubsection{Actions}
		When a regiment has its turn they are allowed to do some actions, Attack, MoveTowards or MoveAway. \\
		For describing these action we set a precondition(what is required to do the action), a failure(what happens if the condition isn't met)
		and result(what happens if the condition is met.
		
		\subsubsection{Attack action}
		Usage: \\
		Attack(Regiment) \\
		
		Precondition: \\
		The regiment has its turn. \\
		For a regiment of attacktype ranged the regiment must be in range. \\
		For a regiment of attacktype melee the enemy regiment must be 1 tile away(diagonals included). \\
		
		Failure: \\
		Nothing happens \\
		
		Result: \\
		The enemy regiment loses Health equivalent to the Damage of the attacking regiment. If the Health reaches 0 the Size of the regiment
		will be decreases by 1 and the Health will be reset. If any Damage is remaining the same procedure is applied to the new Health.
		
		\subsubsection{MoveTowards action}
		Usage: \\
		MoveTowards(Regiment) \\
		
		Precondition: \\
		The regiment has its turn, the given regiment exists and the tiles (left,right,up down) 
		which brings the regiment towards the target regiment isn't blocked.
		
		Failure: \\
		Nothing happens \\
		
		Result: \\
		The regiment will move 1 tile (left,right,up down) towards the friendly/enemy regiment and movement is decreased by 1.
		
		\subsubsection{MoveTowards action}
		Usage: \\
		MoveTowards(Regiment) \\
		
		Precondition: \\
		The regiment has its turn, the given regiment exists and the tiles (left,right,up down) 
		which brings the regiment towards the target regiment isn't blocked.
		
		Failure: \\
		Nothing happens \\
		
		Result: \\
		The regiment will move 1 tile (left,right,up down) towards the friendly/enemy regiment and movement is decreased by 1.
				
		\subsubsection{Terms in the config file}
%		The simulation will need basic instructions which is a common limitation of both teams, to ensure even 'armies'. 
%		As mentioned earlier this is defined in a configuration file.
		
		\subsubsection{Width}
		Width defines the width of the grid
		
		\subsubsection{Height}
		Height defines the height of the grid
		
		\subsubsection{Rules block}
		This block will contain the basic rules and boundaries of the simulation, 
		like limitation of number of teams and/or regiments
		
		\subsubsection{Standards block}
		This block will contain standard stats for regiments. These will be used if the necessary 
		information for a regiment is not entered in the main script, the regiment will use these standard stats.
		
		\subsubsection{Maxima block}
		This block sets the limitations for the stats of the regiments. E.g. regiments must be of size 200 or less.

		