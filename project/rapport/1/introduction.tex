This first chapter is a introduction to the report, here the purpose, problem statement and the outline of the rest of the report will be discussed.

\section{Purpose}
% Purpose sektionen er en forts�ttelse af introduction der fort�ller kort hvad meningen er med projektet, som s� leder over i den mere pr�cise og korte definition i problem statement %
The basic purpose of the project is to design a language which serves some meaningful purpose. The purpose of the language in itself is not to do anything revolutionary in regards of language design, but rather to plug in to a hosting application. The script will then alter the flow of the hosting application entirely dependant upon the script set down by a programmer.
To provide a meaningful context for this language implementation we would create a simulation 'game'. 
Strategy games are one of the most popular genre within the modern game-industry. In strategy games, your victory or defeat depends primarily on the applied tactics. However, most modern strategy games leave room for error through inferior control or attention on the battlefield. The purpose of our language is to define your tactic in advance and have the game play out as predicted by the best abilities of the programmers. As such the main and overall purpose is the implementation of a minor language.

\section{Problem Statement}
During the course of this project our goals were to:

\begin{itemize}
\item Apply concepts of programming language design
\item Determine important design choices and argument for these choices
\item Provide a meaningful context for an implementation of a simple language
\item Implement our own interpreter or compiler
\end{itemize}


\subsection{Outline of the report}
In the following, we give an overview of the structure of the report.

\begin{itemize}

\item \textbf{Chapter 2} describes the problem domain, \\ 
a script-able game that simulates warfare. Two or more competing armies are fighting in a simulated world.\\
In this Chapter we describe the game in details, giving an internal semantics to the possible actions and states(situations) that can occur during the game.
\item \textbf{Chapter 3} presents our high level design choices with respect to   the language that constitutes the 'core' of our system. 
\item \textbf{Chapter 4} presents the language in detail from the point of view of a programmer that needs to program a regiment for our system. The specification of the language, is presented, documenting the principle elements of the language.

%5. The actual design of the language, BNF, EBNF, Contextual analysis.
\item \textbf{Chapter 5} describes the actual design of the language \textit{WAR} representing it in a formal way with the known Backus-Naur Form, and later the Extended Backus-Naur Form. This chapter also represents how the scope-rules and type-rules of the language works.

%6. Implementation (Description of how we implemented it) - Visitor pattern (Page 320, Brown). - Why we chose C# (Maybe tombstone)
\item \textbf{Chapter 6} this chapter is about the implementation of all the different parts of the language. It describes in details how the 4 parts; scanner, parser, contextual analyzer and the interpretation are implemented.

%7. Usecases - (Example of the game (scripts, configs, screenshots etc.))
\item \textbf{Chapter 7} this chapter is about tests of the simulator. Scripts for the simulator and screenshots from the simulator are shown here.

%8. How to expand - (What we would have done if we had infinite time)
\item \textbf{Chapter 8} this chapter represents the ideas that haven't been made, but could have been fulfilled in later projects.

%9. Conclusion - Dicussion
\item \textbf{Chapter 9} this part is a conclusion and discussion part, where we discuss what went good, and what went wrong. This is also where we conclude on the problems we might have experienced.





\end{itemize}


