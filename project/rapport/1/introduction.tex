%Some BULLSHIT INTRODUCTION !!
Here will be an introduction to the project, which will be completed near the end of the project. As is tradition with introductions.
This is certainly not breaking with tradition.

\section{Purpose}
% Purpose sektionen er en forts�ttelse af introduction der fort�ller kort hvad meningen er med projektet, som s� leder over i den mere pr�cise og korte definition i problem statement %
Strategy games is one of the more popular genres within the modern game-industry today. In strategy games, your victory or defeat depends primarily on the applied tactics. However, most modern strategy games leave room for error through inferior control or attention on the battlefield. Though this room for errors often make games more dynamic, as errors give opponents a chance to make a comeback, there doesn't seem to be any games which rewards a superior strategy. In order to determine if strategy games, with no chance of a comeback, would be interesting to play, a simple simulation engine with a related scripting language would suffice to experiment on.

\section{Problem Statement}
During the course of this project our goals were to:


\begin{itemize}
\item Apply concepts of programming languages design
\item Understand decision-making in the design-phase of a programming language
\item Provide a meaningful context for an implementation of a simple language
\item Implement our own interpreter or compiler
\end{itemize}

%Mr Paolos tekst

\subsection{Outline of the report}



%Done Mr Paolo tekst

In the following, we give an overview of the structure of the report.


\begin{itemize}

\item \textbf{Chapter 2} describes the problem domain, \\ a script-able game that simulates warfare. Two or more competing armies are fighting in a simulated world; their actions are based on the execution of the programs.\\In this Chapter we describe the game in details, giving an internal semantics to the possible actions and states(situations) that can occur during the game.
\item \textbf{Chapter 3} presents our high level design choices with respect to   the language that constitutes the 'core' of our system. 
\item \textbf{Chapter 4} presents the language in detail from the point of view of a programmer that needs to program a regiment for our system. The specification of the language, is presented, documenting the principle elements of the language.

%5. The actual design of the language, BNF, EBNF, Contextual analysis.
\item \textbf{Chapter 5} describes the actual design of the language \textit{WAR} representing it in a formal way with the known Backus-Naur Form, and later the Extended Backus-Naur Form. This chapter also represents how the scope-rules and type-rules of the language works.

%6. Implementation (Description of how we implemented it) - Visitor pattern (Page 320, Brown). - Why we chose C# (Maybe tombstone)
\item \textbf{Chapter 6} this chapter is about the implementation of all the different parts of the language. It describes in details how the 4 parts; scanner, parser, contextual analyzer and the interpretation are implemented.

%7. Usecases - (Example of the game (scripts, configs, screenshots etc.))
\item \textbf{Chapter 7} this chapter is all about the use cases, here you can read about how the script files look like, and how you also can build a \textit{config.cfg} files

%8. How to expand - (What we would have done if we had infinite time)
\item \textbf{Chapter 8} this chapter represents the ideas that haven't been made, but could have been fulfilled in later projects.

%9. Conclusion - Dicussion
\item \textbf{Chapter 9} this part is a conclusion and discussion part, where we discuss what when good, and what went wrong. This is also where we conclude on the problems we might have experienced.





\end{itemize}


