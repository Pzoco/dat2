%Description - Contains A description of how to make to script files battle each other, 

\section{ Description of WAR }

	\subsection{The simulation}
		To run a simulation on the WAR simulator a config file and at least two team files have to feed to the simulator. 
		The files will all be written in the WAR language.
		
		The basic rules of the game are defined in the configuration file. 
		This configuration file may be worked out in cooperation by two participants, each representing their own team, 
		to establish ground rules for the war simulation. Such configurations should be, but not limited to, 
		the size of the grid the simulation is going to take place on, how many units each team may own, and when or how the game ends.
		The thought of this is to allow for some flexibility in altering the simulation from game to game.
		Each participant will make a script file they will give to the simulator, 
		which then will interpret the script and simulate the battle. 
		These script files contain how many units and what kind of units the participant would like to use. 
		They may also contain information regarding the behaviour of specific regiments. 
		Too ensue that the battle between two scripts will be fair certain limits on the regiments are specified in the config file.
	
		The one who made the best script will win the game, so every participant has to make the best script, 
		and place the units in the best location on the grid.
	
		When both players are coding their scripts, they will provide the script-files with information about their regiments, 
		but if the player chooses to leave the information out, or even forgets to specify it, 
		the simulator will assign a default value. 
		This helps both rookie scripters and forgetful scripters to provide the simulator 
		with the amount of information required to initiate the simulation.

	\subsection{The language}
		The WAR language will be a simple imperative scripting language. 
		We want the language to be easy to program, such that inexperienced programmers will be able to make a simulation, and that is why we chose to
		make the language very simple. The imperative paradigm is what we felt best supported the idea of a simple programming language. We don't need the
		powerful abstractions of object orientated programming nor the task solving capabilities of the functional languages.
		
		 
		