\section{Implementation}

\begin{figure}[H]
\centering
\includegraphics[scale=1]{rapport/6/figures/parser}
\label{fig:parser}
\caption{Parser}
\end{figure}

\subsubsection{Getting the data}
Inside the simulator figure \ref{fig:simulator} we implemented a $GameDataRetriever$ class which retrieves data from the Decorated Abstract Syntax Tree[DAST], so we have reused the visitor pattern to run through the tree and collect all the data needed to run the simulator. The $GameDataRetriever$ saves the data in the GameClasses which contains Regiment, Grid, Team and Tile- classes. These classes contains all the useful data for the simulator to run.
\subsubsection{Validating the data}
After all the data have been retrieved from the DAST we have to validate if the data is correct. By correct we mean that e.g. in the config.cfg file a $Maxima$ is present, and if the teamfiles.war contains larger numbers than the $Maxima$ requires, there will be an error. The config.cfg file contains standards for which the programmers teamfiles.war will be assigned to if they haven't assigned their own constants. An example: if the teamfile.war does contain a regiment with some size and a name of the regiment, but nothing more, the regiment will then be assigned the default/standard values, if the standard values is nowhere to be found in the config.cfg file there will be an error as well.

\begin{figure}[H]
\centering
\includegraphics[scale=1]{rapport/6/figures/contextual_analyzer}
\label{fig:contextual_analyzer}
\caption{Contextual Analyzer}
\end{figure}

\begin{figure}[H]
\centering
\includegraphics[scale=1]{rapport/6/figures/simulator}
\label{fig:simulator}
\caption{Simulator}
\end{figure}


\subsection{Simulation}

After getting the data and validating it, we are now able to start the simulator with the required data. 
The simulator reads and modifies a $GameState$ in every iteration or turn of the game. The $GameState$ contains the present formation of the game, where all the units are and how large the grid size is. In every iteration the interpreter updates the $GameState$, the interpreter computes the $Behaviour$ from the teamfile.war script. Everytime a unit comes close to another unit and attacks, the interpreter computes how much damage is made, and in the end picks a winner. The output to the screen is made simple, and with the help of the Microsoft XNA \cite{XNA} Framework we have a simple simulation that can show several regiments behave differently and attacking each other.



%Interpreter eksekver en behaviour





