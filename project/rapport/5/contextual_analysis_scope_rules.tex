\section{Semantics}
\label{sec:conanal}
This section will define the scope rules and type rules of WAR.
	\subsection{Scope rules}
		Scope rules dictate the accessibility of identifiers from one scope to another.
		A scope placed inside another scope is called a {\it nested scope}. A {\it scope level} denotes how nested a given scope is,
		e.g. identifiers at scope level 0 is outside any scope, identifiers at scope level 1 is inside one scope etc.
		\begin{lstlisting}[basicstyle=\small\sffamily,
			keywords={break,case,const,continue,default,else,enum,
			for,if,return,switch,while,do,long,void,int,float,double,
			char,struct,typedef,include,size\_t},
			keywordstyle={\color{blue}},
			comment={[l]{//}}, morecomment={[s]{/*}{*/}}, commentstyle=\itshape,
			columns={[l]flexible}, numbers=left, numberstyle=\tiny,
			frameround=fftt, frame=shadowbox, captionpos=b,
			caption={Scope levels}]	
//Scope level 0
Scope
{
	//Scope level 1
	Scope
	{
		//Scope level 2
	}
}
		\end{lstlisting}
		
		WAR has a nested block-structure which means that it allows more than 2 levels of nested scoping. 
		The following scope rules applies for nested block-structures \cite{SPOBOG}.
			
	\begin{itemize}
	\item No identifier may be declared more than once within the same block (at the same level) %SPO
	\item For any applied occurrence there must be a corresponding occurrence, either within the same block or block which is higher up in the nesting. 
	\end{itemize}
		
\newpage
	
