\documentclass{article}
\usepackage{a4wide}
\usepackage{listings,color}
\usepackage[lastexercise]{exercise}
\renewcommand{\ExerciseListHeader}{ 
\textbf{Exercise} \ExerciseHeaderNB:\quad} 
\renewcommand{\QuestionNB}{{\bf Q}:\quad}
\renewcommand{\AnswerListHeader}{ \textbf{A}:\quad} 
\pagestyle{empty}

\title{MVP Assignment 2}
\date{Due date: 1/4/2011}
\begin{document}
\maketitle

% Comment this for answering.
%\newcommand{\question}[1]{#1}
%\newcommand{\answer}[1]{}

% Uncomment this for answering.
\newcommand{\question}[1]{}
\newcommand{\answer}[1]{{#1}}

%
% Write your group room and group number here.
%
\answer{
\begin{flushleft}
{\bf Group room:} 2.1.57\\
{\bf Group number:} D204a
\end{flushleft}
}

\section{Question}

\begin{ExerciseList}
\Exercise Exercise in the book.
\Question \emph{Exercise 3, chapter 6.}
\Answer It is important that the operations are atomically, because if they were not atomically, two thread could lock or unlock the Mutex simultaneously and thus run their routines simultaneously.

\end{ExerciseList}

\section{Hello World}

\begin{ExerciseList}
\Exercise Hello world with threads.
\question{Your program
\begin{itemize}
\item starts $n$ threads, $n$ is given as a command line argument,
\item has every thread print a message of the form
\texttt{Thread i/n: Hello world!} where \texttt{i} is the ID of the
thread and \texttt{n} is the total number of threads,
\item waits for all threads to terminate before terminating.
\end{itemize}
You are not allowed to use any global variable for that. The point of
the exercise is to learn how to prepare data and send them to new
threads without race.
}
\newpage
\Question \emph{Write a ``hello world'' program using pthreads.} 

% Your code goes there.
\begin{lstlisting}[basicstyle=\small\sffamily,
keywords={break,case,const,continue,default,else,enum,
for,if,return,switch,while,do,long,void,int,float,double,
char,struct,typedef,include,size\_t},
keywordstyle={\color{blue}},
comment={[l]{//}}, morecomment={[s]{/*}{*/}}, commentstyle=\itshape,
columns={[l]flexible}, numbers=left, numberstyle=\tiny,
frameround=fftt, frame=shadowbox, captionpos=b,
caption={Hello world with pthreads.},
label=LST:hello]
typedef struct whatever
{
	int id;
	int total;
} storetype;

void helloworld (void *input)
{
	storetype *t=input;
	int id=(*t).id+1;
	int total=t->total;
	printf("Thread %d / %d: Hello World! \n",id,total);
	return NULL;
}


/* start_thread and join_threads functions here */

int main (int argc, char* argv[])
{
	int i;
	int total = atoi(argv[1]);
	pthread_t pths[total];
	storetype storage[total];
	for(i=0;i<total;i++)
	{
		storage[i].id=i;
		storage[i].total=total;
		start_thread(&pths[i],helloworld,&storage[i]);
	}
	join_threads(pths,total);
	exit(EXIT_SUCCESS);
}
\end{lstlisting}

\question{
\textbf{Note:} You have a \texttt{sample-code.c} file where you
will find basic functions to bootstrap you on pthreads.
}

\Exercise $[$Optional$]$ Additional exercise in the book.
\Question $[$Optional$]$ \emph{Exercise 1, chapter 6.}
\Answer .. % Your optional answer here.

\end{ExerciseList}
\newpage
\section{Mandelbrot's Set}

\begin{ExerciseList}
\Exercise
\answer{Parallel Mandelbrot's set generator.}
\question{
In this exercise you parallelize a generator of fractals, here 
Mandelbrot's set. The image is computed as follows: Every point on the
picture corresponds to a point in the complex space. We compute the
following series of numbers: $z_0 = 0$, $z_{n+1} = z_n^2 + c$ where
$c$ is the constant corresponding to the current coordinate. The color
$n$ (here modulo 256) is assigned to the pixel whenever $|z|>K$ with $K$
being some constant value. You can read more on wikipedia if you wish.
}

\Question \emph{Why is this problem a problem trivial to parallelize?}
\Answer None of the data is dependant on each other. Since every pixel on the image can be calculated at any given time and placed at the proper coordinate, interleaving is not necessary.

\Question \emph{Parallelize it with pthreads using a 1-D partitioning on
the output data.}
\question{Your program should use automatically the number of
physical processors you have. Run this on a multi-core machine for the
experiment to be meaningful. Your partitioning should allocate a block
of lines to every processor. You will find some helpful code in
\texttt{sample-code.c}.}

% Your code goes here.
\begin{lstlisting}[basicstyle=\small\sffamily,
keywords={break,case,const,continue,default,else,enum,
for,if,return,switch,while,do,long,void,int,float,double,
char,struct,typedef,include,size\_t},
keywordstyle={\color{blue}},
comment={[l]{//}}, morecomment={[s]{/*}{*/}}, commentstyle=\itshape,
columns={[l]flexible}, numbers=left, numberstyle=\tiny,
frameround=fftt, frame=shadowbox, captionpos=b,
caption={Parallelized computation of Mandelbrot's set.},
label=LST:mandelbrot1]
pthread_t threadcount[count_cpus()];
size_t cpucount = count_cpus();


void *startcomputation(void* coreid)
{
    size_t thread_id = (size_t)coreid;
	printf("Thread %d starting \n", thread_id);
    size_t x,y;
	for (y=(height/cpucount)*thread_id;y<(height/cpucount)*(thread_id +1); y++)
	{
		for (x=0;x<width;x++)
		{
			compute(x, y, &data);
		}
	}
	printf("Thread %d finished work \n", thread_id);
}

size_t mkthrcount;
for (mkthrcount=0;mkthrcount<cpucount;mkthrcount++)
{
	pthread_create(&threadcount[mkthrcount], NULL, startcomputation, (void*)mkthrcount);
}
for (mkthrcount=0;mkthrcount<cpucount;mkthrcount++)
{
	pthread_join(threadcount[mkthrcount], NULL);
}
\end{lstlisting}

\Question
\question{
When you benchmark this (with the provided script) you will see some
disappointing results.}
\emph{First, explain the different outputs of the time command, in
  particular why user time is greater than real time.}
\Answer Real time: The time elapsed since we initiated the application until work is finished. This corresponds with the time that passes if you were to time it with a watch. \newline
User time: The time the processor spends working in the application. In thise case since the application is multithreaded, and running on a processor with 4 cores, the User time should be roughly equivalent to a fourth of the real time. \newline
Sys time: This is the time spent doing system calls, such as opening files, saving files etc.

\Question
\question{
The seamingly trivial-to-parallelize program suffers from some very
important issue.} \emph{What is the problem and where does it come from?}
\question{Hint: The next question solves ``the problem''.}
\Answer The problem is that while we expected our Real time to go down to a quarter of the User time, we're only seeing it at about half. The main issue with this is that not all four cores have as much work to do, some parts of the image require a lot more computation than others, which means that the workload is not evenly distributed among the cores using this method.

\Question
\emph{Parallelize it again using a 1-D round-robin partitioning on the
  output data.}
\question{The round robin partitioning assigns each row to a
different processor at regular turns. For this exercise, the
modification to the previous version should be minor.}

% Your code goes there.
\begin{lstlisting}[basicstyle=\small\sffamily,
keywords={break,case,const,continue,default,else,enum,
for,if,return,switch,while,do,long,void,int,float,double,
char,struct,typedef,include,size\_t},
keywordstyle={\color{blue}},
comment={[l]{//}}, morecomment={[s]{/*}{*/}}, commentstyle=\itshape,
columns={[l]flexible}, numbers=left, numberstyle=\tiny,
frameround=fftt, frame=shadowbox, captionpos=b,
caption={Parallelized computation of Mandelbrot's set.},
label=LST:mandelbrot2]
#define LINESYNC if(y % cpucount == thread_id)
void *startcomputation(void* coreid)
{
    size_t thread_id = (size_t)coreid;
	printf("Thread %d starting \n", thread_id);
    size_t x,y;
	for (y=0; y<height;y++)LINESYNC
	{
		for (x=0;x<width;x++)
		{
			compute(x, y, &data);
		}
	}
	printf("Thread %d finished work \n", thread_id);
}
\end{lstlisting}
Note: Test times are now roughly a quarter.
\end{ExerciseList}
\newpage
\section{Parallel Matrix Multiplication}

\begin{ExerciseList}
\Exercise
\answer{Parallel matrix multiplication by block.}
\question{
Your \texttt{pmatrix.c} from assignment 1 has been updated in this
assignment. First, copy your block matrix multiplication from
assignment 1.}
\Question
\emph{Write the job function for every thread to compute the
  multiplication by block.}
\question{The function
is already here and is called \texttt{job\_block\_mat\_mult}.
Hint: Use the macros DECOMPOSE\_BY\_ROW and DECOMPOSE\_BY\_BLOCK to
ease the programming. This is an easy method. Keen students may
rewrite their functions to avoid if-statements at every loop.}

% Your code goes here.
\begin{lstlisting}[basicstyle=\small\sffamily,
keywords={break,case,const,continue,default,else,enum,
for,if,return,switch,while,do,long,void,int,float,double,
char,struct,typedef,include,size\_t},
keywordstyle={\color{blue}},
comment={[l]{//}}, morecomment={[s]{/*}{*/}}, commentstyle=\itshape,
columns={[l]flexible}, numbers=left, numberstyle=\tiny,
frameround=fftt, frame=shadowbox, captionpos=b,
caption={Parallelized matrix multiplication by block.},
label=LST:blockmatmult]
void* job_block_mat_mult(matmult_data_t *data)
{
    double *a = data->a;
    double *b = data->b;
    double *c = data->c;
    size_t n = data->n;
    size_t id = data->id;
    size_t bi, bj, bk, i, j, k, maxi, maxj, maxk;

    for(bi = 0;bi<n;bi+=BLOCK)DECOMPOSE_BY_ROW
    {
        maxi = min(bi+BLOCK,n);
        for(bj = 0;bj<n;bj+=BLOCK)DECOMPOSE_BY_BLOCK
        {
            maxj = min(bj+BLOCK,n);
            for(bk = 0;bk < n;bk+=BLOCK)
            {
                maxk = min(bk+BLOCK,n);
                for(i = 0;i<maxi;i++)
                {
                    for(j= 0;j<maxj;j++)
                    {
                        double temp = 0;
                        for(k = 0;k<maxk;k++)
                        {
                        temp+= a[i*n+k]*b[k*n+j];
                        }
                        c[i*n+j] +=temp;
                    }
                }
            }
        }
    }
}
\end{lstlisting}

\end{ExerciseList}


\newpage
\section{Authors}
I/We have solved these exercises independently, and each of us has actively
participated in the development of all of the exercise solutions.
\vspace{1cm}

\noindent
\begin{tabular}{p{70mm}p{70mm}}

%
% Replace ``Name x'' by your name.
%

Mads Vestergaard Carlsen & Rasmus S$\emptyset$gaard Jacobsen \\
\dots\dotfill\dots & \dots\dotfill\dots \\
 &  \\
& \\
& \\

Dan Duus Th$\emptyset$isen &  \\
\dots\dotfill\dots &  \\
&  \\
& \\
& \\
\end{tabular}

\end{document}
