\documentclass{article}
\usepackage{a4wide}
\usepackage{listings,color}
\usepackage[lastexercise]{exercise}
\renewcommand{\ExerciseListHeader}{ 
\textbf{Exercise} \ExerciseHeaderNB:\quad} 
\renewcommand{\QuestionNB}{{\bf Q}:\quad}
\renewcommand{\AnswerListHeader}{ \textbf{A}:\quad} 
\pagestyle{empty}

\title{MVP Assignment 4 - MPI}
\date{Due date: 28/4/2010}
\begin{document}
\maketitle

% Leave this commented for answering.
 \newcommand{\question}[1]{#1}
 \newcommand{\answer}[1]{}

% Leave this uncommented for answering.
%\newcommand{\question}[1]{}
%\newcommand{\answer}[1]{{#1}}

%
% Write your group room and group number here.
%
\answer{
\begin{flushleft}
{\bf Group room:} FIXME\\
{\bf Group number:} FIXME
\end{flushleft}
}

\section{OpenMP}

The idea is to parallelize the matrix
multiplication with OpenMP in a very easy way.

\begin{ExerciseList}
\Exercise Copy \texttt{pmatrix.c} from your previous assignment.

\Question \emph{Parallelize the rearranged and the block-matrix
  multiplications using OpenMP.}

\end{ExerciseList}

% Your code goes here.
\begin{lstlisting}[basicstyle=\small\sffamily,
keywords={break,case,const,continue,default,else,enum,
for,if,return,switch,while,do,long,void,int,float,double,
char,struct,typedef,include,size\_t},
keywordstyle={\color{blue}},
comment={[l]{//}}, morecomment={[s]{/*}{*/}}, commentstyle=\itshape,
columns={[l]flexible}, numbers=left, numberstyle=\tiny,
frameround=fftt, frame=shadowbox, captionpos=b,
caption={Your OpenMP implementation.},
label=LST:openmp]

/* Your relevant code goes here. */
\end{lstlisting}

\section{MPI}

\begin{ExerciseList}
\Exercise Warm-up with a hello-world program.
\Question \emph{Complete the hello-world program so that every process
  prints its rank and the size of the (world) communicator.}
\end{ExerciseList}

% Your code goes here.
\begin{lstlisting}[basicstyle=\small\sffamily,
keywords={break,case,const,continue,default,else,enum,
for,if,return,switch,while,do,long,void,int,float,double,
char,struct,typedef,include,size\_t},
keywordstyle={\color{blue}},
comment={[l]{//}}, morecomment={[s]{/*}{*/}}, commentstyle=\itshape,
columns={[l]flexible}, numbers=left, numberstyle=\tiny,
frameround=fftt, frame=shadowbox, captionpos=b,
caption={Your hello-world implementation.},
label=LST:hello]
int main(int argc, char *argv[])
{

int rank, size;

MPI_Init(&argc, &argv); // Init
MPI_Comm_size(MPI_COMM_WORLD, &size); // Get the size
MPI_Comm_rank(MPI_COMM_WORLD, &rank); // Get the rank
printf("Process number %d of a total of %d says HELLO WORLD! \n", rank, size); // Hello World
MPI_Finalize();              // Finalize

return 0;

}

\end{lstlisting}

\begin{ExerciseList}
\Exercise \question{You will experiment blocking and non-blocking
  communications and deadlock issue. The goal is to have all processes
  starting with sending to the next process in the ring its own rank
  (times $n$). Then the processes forward the messages around $size$
  times.}
\answer{Blocking communication.}
  \Question \emph{Complete the cycle-mpi program with MPI\_Send.}

\Exercise
 \question{As it turns out, the OpenMPI manual states ``This  routine
  will  block until the message is sent to the destination.'' whereas
  the LAM manual says `` This  function  may  block until the message
  is received. Whether or not MPI\_Send blocks depends on factors such
  as how large the message is, how many messages are pending to the
  specific destination, whether LAMD or C2C communication is being
  used, etc.''.}
\answer{Implementation features.}
  \Question
  \emph{Experiment with $n$ to see when the sending
    becomes really blocking and find the deadlock.}

  \Question \emph{Fix the deadlock by breaking the cycle-dependency.}

  \Exercise
\question{It is possible to break the deadlock by using non-blocking
  communication instead.}
\answer{Non-blocking communication.}
  \Question \emph{Fix the deadlock by using the non-blocking MPI\_Isend.}

\end{ExerciseList}

% Your code goes here.
\begin{lstlisting}[basicstyle=\small\sffamily,
keywords={break,case,const,continue,default,else,enum,
for,if,return,switch,while,do,long,void,int,float,double,
char,struct,typedef,include,size\_t},
keywordstyle={\color{blue}},
comment={[l]{//}}, morecomment={[s]{/*}{*/}}, commentstyle=\itshape,
columns={[l]flexible}, numbers=left, numberstyle=\tiny,
frameround=fftt, frame=shadowbox, captionpos=b,
caption={Cycle 1: Blocking, deadlock.},
label=LST:cycle1]
/* Your relevant code goes here. NOT the whole file. */
\end{lstlisting}

% Your code goes here.
\begin{lstlisting}[basicstyle=\small\sffamily,
keywords={break,case,const,continue,default,else,enum,
for,if,return,switch,while,do,long,void,int,float,double,
char,struct,typedef,include,size\_t},
keywordstyle={\color{blue}},
comment={[l]{//}}, morecomment={[s]{/*}{*/}}, commentstyle=\itshape,
columns={[l]flexible}, numbers=left, numberstyle=\tiny,
frameround=fftt, frame=shadowbox, captionpos=b,
caption={Cycle 2: Blocking, no deadlock.},
label=LST:cycle2]
/* Your relevant code goes here. NOT the whole file. */
\end{lstlisting}

% Your code goes here.
\begin{lstlisting}[basicstyle=\small\sffamily,
keywords={break,case,const,continue,default,else,enum,
for,if,return,switch,while,do,long,void,int,float,double,
char,struct,typedef,include,size\_t},
keywordstyle={\color{blue}},
comment={[l]{//}}, morecomment={[s]{/*}{*/}}, commentstyle=\itshape,
columns={[l]flexible}, numbers=left, numberstyle=\tiny,
frameround=fftt, frame=shadowbox, captionpos=b,
caption={Cycle 3: Non-blocking, no deadlock.},
label=LST:cycle3]
/* Your relevant code goes here. NOT the whole file. */
\end{lstlisting}



\newpage
\section{Authors}
I/We have solved these exercises independently, and each of us has actively
participated in the development of all of the exercise solutions.
\vspace{1cm}

\noindent
\begin{tabular}{p{70mm}p{70mm}}

%
% Replace ``Name x'' by your name.
%

Name 1 & Name 2 \\
\dots\dotfill\dots & \dots\dotfill\dots \\
Signature & Signature \\
& \\
& \\

Name 3 & Name 4 \\
\dots\dotfill\dots & \dots\dotfill\dots \\
Signature & Signature \\
& \\
& \\

Name 5 & Name 6 \\
\dots\dotfill\dots & \dots\dotfill\dots \\
Signature & Signature \\
& \\
& \\

Name 7 & Name 8 \\
\dots\dotfill\dots & \dots\dotfill\dots \\
Signature & Signature \\
& \\
& \\
\end{tabular}

\end{document}
