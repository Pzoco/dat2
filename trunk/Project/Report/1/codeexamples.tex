\section{Code Examples}
	In this section we will show how the scripts written in the WAR scripting language will look like.
	Normally this section will only contain one code example or more code examples of the same type, 
	but because we are quite uncertain how the script could like we have included 3 types of scripts.
	\subsection{Code example 1}
		\subsection{Elements}
			This code 
		\subsection{Code}
			\begin{lstlisting}[basicstyle=\small\sffamily,
				keywords={break,case,const,continue,default,else,enum,
				for,if,return,switch,while,do,long,void,int,float,double,
				char,struct,typedef,include,size\_t},
				keywordstyle={\color{blue}},
				comment={[l]{//}}, morecomment={[s]{/*}{*/}}, commentstyle=\itshape,
				columns={[l]flexible}, numbers=left, numberstyle=\tiny,
				frameround=fftt, frame=shadowbox, captionpos=b,
				caption={Code example 1},
				label=LST:code1]
#begin

// The map is a 50x50 grid

//Instantiates two regiments one of 5 Archer units and one of 7 Pikemen units:
AddRegiment(5, Archer);
AddRegiment(7, Pikemen);

//Places the archer regiment from coordinate 1,1 to 5,1
//Places the pikemen regiment from coordinate 1,3 to 7,3
Placement[(1,1),(5,1),(Archer)][(1,3),(7,3),(Pikemen)]

//Checks in the archer enemy detects some units
if ((Regiment(Archer) spotsUnits())
{
	//Attack the units it spotted
	AttackSpottedUnits();

	//If the regiment it self gets attacked it retreats
	if((Regiment(Archer) getsAttacked())
	{
		Retreat();
	}
}

//Checks in the archer enemy detects some units
if ((Regiment(Pikemen) !spotsUnits())
{
	//Moves the regiment forward	
	moveForward();
	if((Regiment(Pikemen)) getsAttacked())
	{
		GettingAttackedCounterAttack();
	}
}


//If either regiment reaches the end of the grid turn around
if (Regiment(Archer),(Pikemen)) reachEnd();
{
	turnArround();
}

#end

		 	\end{lstlisting}

	\subsection{Code example 2}
	\subsection{Code example 3}