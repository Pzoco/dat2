\section{Code Examples}
	In this section we will show how the scripts written in the WAR scripting language will look like.
	Normally this section will only contain one code example or more code examples of the same type, 
	but because we are quite uncertain how the script could like we have included 3 types of scripts.
	\subsection{Code example 1}
		\subsection{Elements}
			First code example shows two regiments and the regiments interacts. The code starts with a $\#begin$ and ends with a $\#end$.
		\subsection{Code}
			\begin{lstlisting}[basicstyle=\small\sffamily,
				keywords={break,case,const,continue,default,else,enum,
				for,if,return,switch,while,do,long,void,int,float,double,
				char,struct,typedef,include,size\_t},
				keywordstyle={\color{blue}},
				comment={[l]{//}}, morecomment={[s]{/*}{*/}}, commentstyle=\itshape,
				columns={[l]flexible}, numbers=left, numberstyle=\tiny,
				frameround=fftt, frame=shadowbox, captionpos=b,
				caption={Code example 1},
				label=LST:code1]
#begin

// The map is a 50x50 grid

//Instantiates two regiments one of 5 Archer units and one of 7 Pikemen units:
AddRegiment(5, Archer);
AddRegiment(7, Pikemen);

//Places the archer regiment from coordinate 1,1 to 5,1
//Places the pikemen regiment from coordinate 1,3 to 7,3
Placement[(1,1),(5,1),(Archer)][(1,3),(7,3),(Pikemen)]

//Checks in the archer enemy detects some units
if ((Regiment(Archer) spotsUnits())
{
	//Attack the units it spotted
	AttackSpottedUnits();

	//If the regiment it self gets attacked it retreats
	if((Regiment(Archer) getsAttacked())
	{
		Retreat();
	}
}

//Checks in the archer enemy detects some units
if ((Regiment(Pikemen) !spotsUnits())
{
	//Moves the regiment forward	
	moveForward();
	if((Regiment(Pikemen)) getsAttacked())
	{
		GettingAttackedCounterAttack();
	}
}


//If either regiment reaches the end of the grid turn around
if (Regiment(Archer),(Pikemen)) reachEnd();
{
	turnArround();
}

#end

		 	\end{lstlisting}

	\subsection{Code example 2}
			\subsection{Elements}
				In this example we have two different files. The config file specifies the rules of the game
				such as grid size or allowed sizes of regiments. In the team tile a team is specified. In this file
				a regiment is defined and a hero unit. Inside the regiment file is the stats of the regiment and
				a behaviour block exists, inside the behaviour of the regiment is defined.
			\subsection{Code}
				\subsubsection{Team file}
					\begin{lstlisting}[basicstyle=\small\sffamily,
					keywords={break,case,const,continue,default,else,enum,
					for,if,return,switch,while,do,long,void,int,float,double,
					char,struct,typedef,include,size\_t},
					keywordstyle={\color{blue}},
					comment={[l]{//}}, morecomment={[s]{/*}{*/}}, commentstyle=\itshape,
					columns={[l]flexible}, numbers=left, numberstyle=\tiny,
					frameround=fftt, frame=shadowbox, captionpos=b,
					caption={Team file of code example 3},
					label=LST:code31]
Team "Ninja Monkeys"

Regiment "Silent Monkeys"
//Archers
{
	
	Size = 200;
	Type = Ranged;
	Range = 120;
	Damage = 2;
	Movement = 30;
	AttackSpeed = 1;
	
	//Defining the behaviour of the regiment
	Behaviour
	{
		(Unit/regiment/whatever) enemy = Search_for_Enemies();
		if (enemy.Position <= Range && enemy.Type == Melee)
		{
			Attack.Position(enemy.Position);
		}
		else if (enemy.Position <= Range+Movement)
		{
			while(enemyPosition < Range)
			{
				MoveTowards(enemy.Position)
			}
			Attack.Position(enemy.Position);
		}
		if (enemy.position <= Range && enemy.Type == Hero)
		{
			MoveAway(enemy.position)
			if(enemy.position <= Range)
			{
				Attack.Enemy(enemy);
			}
		}
	}
}

Hero "Old Neon Monkey"
//Hero unit
{
	Type=Hero.Melee;
	Movement = 50;
	Damage = 30
	AttackSpeed = 5;
	Behaviour
	{
		Regiment enemy = Search_for_Enemy();
		if(enemy.Position == MeleeRange)
		{
			Attack.Enemy(enemy);
			Rape(enemy);
		}
		else
		{
			MoveTowards(enemy.Position);
		}
	}
}
					\end{lstlisting}
				\subsubsection{Config file}
					\begin{lstlisting}[basicstyle=\small\sffamily,
					keywords={break,case,const,continue,default,else,enum,
					for,if,return,switch,while,do,long,void,int,float,double,
					char,struct,typedef,include,size\_t},
					keywordstyle={\color{blue}},
					comment={[l]{//}}, morecomment={[s]{/*}{*/}}, commentstyle=\itshape,
					columns={[l]flexible}, numbers=left, numberstyle=\tiny,
					frameround=fftt, frame=shadowbox, captionpos=b,
					caption={Config file of the code example 3},
					label=LST:code32]
Config

//Definition of the grid
Grid "Banana Island"
{
	HorizontalSize = 512;
	VerticalSize = 128;
	ObstaclesDensity = 2; //Goes from 1-10
}

Rules
{
	//Standards defines the standard fields for regiments
	//If a regiment haven't defined for example Type it will use the Type in Standards
	Standards
	{
		Size = 200;
		Type = Melee;
		Behaviour
		{
			//Standard behaviour
		}
	}
	//Rules which the regiments can't exceed
	Maximums
	{
		RegimentSize = 500
		Heros = 1;
		Regiments = 4;
	}
}
					\end{lstlisting}				
	\subsection{Code example 3}
			\subsection{Elements}
				This code is resembles the code in code example 2. The biggest difference is that
				elements found in most normal programming languages are left out, like ; or datatypes
				(int,string).
			\subsection{Code}
				\begin{lstlisting}[basicstyle=\small\sffamily,
				keywords={break,case,const,continue,default,else,enum,
				for,if,return,switch,while,do,long,void,int,float,double,
				char,struct,typedef,include,size\_t},
				keywordstyle={\color{blue}},
				comment={[l]{//}}, morecomment={[s]{/*}{*/}}, commentstyle=\itshape,
				columns={[l]flexible}, numbers=left, numberstyle=\tiny,
				frameround=fftt, frame=shadowbox, captionpos=b,
				caption={Code example 3},
				label=LST:code3]
unit pirate = 10
unit soldier = 5

Pirate
{
	Speed = 4
	AttackPower = 1
	DefensivePower = 2
	
	if(Enemy is near)
	{
		MoveTowards
	}
	else
	{
		wait
	}
}

Soldier
{
	Speed = 1
	AttackPower = 10
	DefensivePower = 5
	if(enemy is inRange)
	{
		Attack
	}
	else if(Enemy is near)
	{
		moveTowards		
	}
}
				\end{lstlisting}				
