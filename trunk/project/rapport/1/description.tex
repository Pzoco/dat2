%Description - Contains A description of how to make to script files battle each other, 

\section{ Description of the WAR\# language }

	The language is intended to be run in a hosting application, the simulation environment. During this simulation our scripting language will be interpreted to the simulation environment, and the simulation will be constructed, and initiated by the script.
	
	The basic rules of the game are defined in the scripting language in a configuration file. This configuration file may be worked out in cooperation by two participants, each representing their own team, to establish ground rules for the war simulation. Such configurations should be, but not limited to, the size of the grid the simulation is going to take place on, how many units each team may own, and when or how the game ends.
	The thought of this is to allow for some flexibility in altering the simulation from game to game.
	Each participant will make a script file they will give to the simulator, 
	which then will interpret the script and simulate the battle. 
	These script files contain how many units and what kind of units the participant would like to use. They may also contain information regarding the behaviour of specific regiments.

	The one who made the best script will win the game, so every participant has to make the best script, 
	and place the units in the best location on the grid.

When both players are coding their scripts, they will provide the script-files with information about their regiments, but if the player chooses to leave the information out, or even forgets to specify it, the simulator will assign a default value. This helps both the rookie scripters and forgetful scripters to provide the simulator with the amount of information required to initiate the simulation.

It is intended to be rather easy to script in this language, although some very basic knowledge of languages will be required.


%\fxnote{}