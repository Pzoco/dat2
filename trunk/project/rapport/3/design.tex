%Intro

\section{Design criteria}
%
%\subsection{Write-ability}
%
%\subsection{Readability}
%\subsection{Reliability}
%\subsection{Orthogonality}
%\subsection{Uniformity}
%\subsection{Maintainability}
%\subsection{Generality}
%\subsection{Extensibility}
%\subsection{Standardability}
%\subsection{Implementability}
%
%Importance: Low Medium High Very high

\begin{table}[H]
	\begin{tabular}{l l p{7cm}}
	Criteria			&	Importance	&		Description	\\		
	Write-ability 		& 	Very high	&		How easy and how fast can the program be written\\
	Readability			& 	High		&		How easy is already written code to read\\
	Reliability			& 	Medium		&		The program wil not behave unexspectively\\
	Orthogonality		& 	Low			&		.\\
	Uniformity			& 	High		&		Similar features should look and behave similar\\
	Maintainability		& 	Very high	&		How fast can an error be found and corrected\\
	Generality			& 	Medium		&		No or few special cases, combine the special cases and construct a more general one\\
	Extensibility		& 	Low			&		Able to add new constructs to the language\\
	Standardability		& 	Medium		&		Able to stransport the language to other computers\\
	Implementability	& 	Low			&		.\\
	\end{tabular}
	\caption{Design criterias}
	\label{tab:criteria_tabular}
\end{table}


Tabular \ref{tab:criteria_tabular} lists an overview over how the design criterias of the WAR language have been made, and we will dig deeper in the criterias with importance high or very high.

The primary focus of this language, has been on write-ability and somewhat readability which is easily fulfilled when write-ability is the main goal. By having the write-ability as a very important choice we think that beginners in programming should be able to write their own script, and the maintainability is also rated very high, so the programmer of the language easy can troubleshoot the script, and find the errors that might have occured and rapidly fix them. The uniformity of the language is of a high importance since this helps the beginners to fastly learn the language and understand how the different part of the language works.


\subsection{Paradigm}
The design criteria of the WAR language is how advanced or how simple it should be, what the main focus area and what sort of paradigm it should be. We have chosen the imperative programming paradigm because in contrast to the object-oriented programming paradigm which is used to program in a more abstract sense, this should be simple and straight forward. The imperative programming paradigm describes computation in terms of statements that can change a program state and in contrast to the functional programming paradigm which treats computation as the evaluation of mathematical functions the imperative paradigm has been chosen to the WAR language since it uses statements, eventhough the user cannot declare statements, but the user can call different statements in the behaviour block. 


