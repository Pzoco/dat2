\subsection{Type rules}
	Type rules regulate the expected types of arguments and 
	types of returned values for the operations of a language \cite{TypeRule}.
	The following types are present in \textit{WAR}: \\
	\begin{itemize}
		\item Integer
		\item Boolean
		\item Position
		\item Regiment
		\item AttackType
	\end{itemize}
	
	\subsubsection{Type rules of \textit{WAR}}
	This type rule is used in expressions or when assigning a constant to a integer value.
	\begin{typerule} 
		$I1$ $ o $ $I2$ is type correct and of type Integer \\
		if $I1$ and $I2$ are type correct and of type Integer \\
		and $o$ is an operator where $o\in \{-,+,/,* \}$
	\end{typerule}
	This type rule is used for checking if conditional statements are type correct.
	\begin{typerule} 
		$C(E)\{D\}$ is type correct\\
		if $E$ is of type Boolean and $D$ is type correct \\
		and $C$ is a control structure where $C \in \{else$ $if, if, while\}$
	\end{typerule}
	This type rule is used for assignments.
	\begin{typerule} 
		$ID = A$ is type correct\\
		if $ID$ and $A$ are type correct and of the same type.
	\end{typerule}
	This type rule is used for {\it Position(x,y)}, which is used when assigning a 
	position to a regiment.
	\begin{typerule} 
		$Position(I1,I2)$ is type correct and of type Position\\
		if $I1$ and $I2$ are type correct and of type Integer
	\end{typerule}
	This regiment is for boolean binary expressions.
	\begin{typerule} 
		$E1$ $ o $ $E2$ is type correct and of type Boolean\\
		if $E1$ and $E2$ are type correct and of type Integer \\
		and $o$ is an operator where $o\in \{ ==,>=,<=,<,> \}$
	\end{typerule}
	
	All these type rules makes the scripting more secure, because if a argument or types of returned values is not evaluated to type correct 
	using these type rules then an error is present. \\
	
	These definitions are the end of the section on semantics of WAR. 
	With the syntax and semantics written a documentation for the language can be written, which will be the topic of the next chapter.