In this chapter the syntax, scope rules and type rules of the WAR language will be defined. The Syntax will be defined by a BNF, which will be extended to an EBNF - both can be found in the Appendix. The scope rules and type rules are defined within this chapter.
\section{Syntax}
	The first step in writing our interpreter was to construct a BNF of our language. 
	BNF is an acronym for {\it Backus Naur form}, which is a way of
	structuring the language in a formal way, to further ease implementation. 
	A BNF consists of a set of {\it production rules} and each
	production rule consists of: \\
	\begin{itemize}
		 \item Terminals: A {\it terminal } is a literal string or character as it appears in the code.
		 \item Non-terminal: A {\it non-terminal } may consist of an arbitrary number of Non-terminals and Terminals, used for structuring the language.
	\end{itemize}
	
	\subsection*{Production Rule}
		A production rule has the form: \\
		\begin{equation}
			N ::= X
		\end{equation}
		
		where $X$ is a Non-terminal or a Terminal. \\
		
		The elements in a production rule can be identified by: \\
		Terminals are written in {\bf bold } and 
		Non-terminals are just written in plain text.
	\subsection*{Grouping of production rules}
		If we have two production rules on the form:\\
		\begin{equation}
			N ::= X 
		\end{equation}
		\begin{equation}
			N ::= Y 
		\end{equation}
		we are allowed to make a transformation: \\
		\begin{equation}
			N ::= X | Y 
		\end{equation}
		Meaning N is X or Y \\
		
	\begin{comment}
	\subsection*{Example of a small BNF}
		\begin{center}
			\begin{tabular}{l l l}
				A		&	::=		&	 AB$ \mid $ B \\
				B		&	::=		&	{\bf b} $\mid$ {\bf n } $\mid$ {\bf f } \\
			\end{tabular}
		\end{center}
		This BNF would allow us to write programs like \\(We use $\rightarrow$ to show applying of a production rule): \\
		{\bf bnf}: A $\rightarrow$ AB $\rightarrow$ A{\bf f} $\rightarrow$ 
		AB{\bf f} $\rightarrow$A{\bf nf} $\rightarrow$ B{\bf nf}$\rightarrow${\bf bnf } \\

	\subsubsection{Terminals}
		\begin{longtable}[l]{l}
			$\{$\\
			$\}$\\
			$>$\\
			$<$\\
			$($\\
			$)$\\
			$=$\\
			$+$\\
			$-$\\
			$*$\\
			$/$\\
			$\&$\\
			$|$\\
			$"$\\
			$,$\\
			$.$\\
			$;$\\
			$Attack$\\
			$AttackSpeed$\\
			$Behaviour$\\
			$Config$\\
			$Damage$\\
			$Distance$\\
			$else$\\
			$Grid$\\
			$Health$\
			$Height$\\
			$if$\\
			$Maxima$\\
			$Melee$\\
			$MoveAway$\\
			$Movement$\\
			$MoveTowards$\\
			$Position$\\
			$Range$\\
			$Ranged$\\
			$Regiment$\\
			$RegimentPosition$\\
			$Regiments$\\
			$SearchForEnemies$\\
			$SearchForFriends$\\
			$Size$\\
			$Standards$\\
			$Team$\\
			$Teams$\\
			$Type$\\
			$While$\\
			$Width$\\
		\end{longtable}
%	\subsubsection{Nonterminals}
%		\begin{tabular}{l c}
%			Behaviour-Block   & Control-Structure   \\
%			Single-Command    & UnitStat 			\\
%			UnitStat-Name	  & Integer-Literal		\\
%			Comment			  & Team-File 			\\
%			Regiment-Block	  & Expression			\\
%			Operator		  & Config-File			\\
%			Grid-Block	      & Rules-Block 		\\
%			Maxima-Block	  & MaximaStat		\\
%			MaximaStat-Name & Standards-Block	 	\\
%			StandardsStat	  & StandardsStat-Name 	\\
%			Identifier		  & 					\\
%		\end{tabular}
	\end{comment}
	\subsubsection{BNF}
		The BNF was made by looking at an example code - a code we wrote ourselves as an example of how we would like the final code to look. This example code can be seen in Listing \ref{LST:code31}. The BNF can be seen Appendix \ref{app:bnf}. \\
		Here is a production rule from the BNF: \\
		\begin{tabular}{l l l}
		Identifier			&	::=	&Letter\\
							&$\mid$	&Identifier Digit\\
							&$\mid$	&Identifier Letter\\
		\end{tabular}
	
	\subsection{EBNF}
		{\it EBNF} is an acronym for {\it Extendend Backus Naur Form} and is, as the name refers to an extension of BNF.
		The EBNF allows us to use regular expressions to express production rules. 
		This makes our rule set more compact and easier to implement.
	
	\subsubsection*{Regular expressions}
		A regular expression is a convenient way of expressing strings. We use the following regular expressions: \\
		\begin{itemize}
			\item *(star): States that the terminal or non-terminal must be used 0 or more times.
			\item Parentheses: Used for grouping of non-terminals and terminals.
			\item $\epsilon$: Represents the empty string.
		\end{itemize}
		Please note that these regular expressions can only be used on the right side of the production rules. \\
		
		A transformation of a BNF is done systematically by applying {\it left factorization }, 
		{\it elimination of left recursion} and {\it substitution of non-terminals }.
		
		\subsubsection*{Left factorization}
			If we have a production rule on the form: \\
			
			\begin{tabular}{l l l}
			\centering
				T		&	::=		&	AB$\mid$AC \\
			\end{tabular}
			
			We can do a left factorization: \\
			
			\begin{tabular}{l l l}
			\centering
				T		&	::=		&	A(B$\mid$C) \\
			\end{tabular}
			
			This can be done because T will always start with an A and end with either a B or C.
			
		\subsubsection*{Elimination of left recursion}
			If we have a production rule on the form: \\
			
			\begin{tabular}{l l l}
			\centering
				T		&	::=		&	A$\mid$TB \\
			\end{tabular}
			
			We can do an elimination of left recursion: \\
			
			\begin{tabular}{l l l}
			\centering
				T		&	::=		&	A(B)* \\
			\end{tabular}
			
			To understand how we can do this look at the first production rule. To terminate the production rule we need 
			to put an A, so T will always consist of an A. When we are not selecting an A(, and terminating) we are only making B's, which is
			the same as B*.
		\subsubsection*{Substitution of non-terminals}
			If we have two production rules on the form: \\
			
			\begin{tabular}{l l l}
			\centering
				T		&	::=		&	U $\mid$ AUC \\
				U		&	::=		&	{\bf ab} $\mid$ {\bf ba} \\
			\end{tabular} \\
			
			We can do a substitution of the non-terminal U for every production rule. This means that we 
			substitute any occurrence of U in a production rule with what is on the right hand side of U like this: \\
			
			\begin{tabular}{l l l}
			\centering
				T		&	::=		&	({\bf ab} $\mid$ {\bf ba}) $\mid$ A ({\bf ab} $\mid$ {\bf ba})B \\
			\end{tabular}
	\subsection{The EBNF of the WAR Language}
		To obtain the EBNF we applied left factorization, elimination of left recursion and substitution of non-terminals on the BNF 
		and can be seen in Appendix \ref{app:ebnf}.
		To illustrate how we transformed our BNF to an EBNF we will look at an example.
		\subsubsection{Transformation of production rule of Identifier}
			Here is the production rule of Identifer: \\
			
			\begin{tabular}{l l l}
				Identifier			&	::=	&Letter\\
									&$\mid$	&Identifier Digit\\
									&$\mid$	&Identifier Letter\\
			\end{tabular} \\
			
			We can first apply Left factorization giving \\ 
			us the new production rule: \\
			
			\begin{tabular}{l l l}
				Identifier			&	::=	&Letter\\
									&$\mid$	&Identifier (Digit $\mid$ Letter)\\
			\end{tabular}\\	
			
			Then we can apply Elimination of left recursion \\
			giving us the new production rule: \\
			
			\begin{tabular}{l l l}
				Identifier			&	::=	&Letter(Digit $\mid$ Letter)*\\
			\end{tabular}	
			