\section{Syntax}
	The first step in writing our interpreter is to make a BNF of our language. BNF stands for {\it Backus Naur form} and is a way of
	structuring the language in a formal way, which will make it easy to implement. A BNF consists of a set of {\it production rules} and each
	production rule consists of: \\
	\begin{itemize}
		 \item Terminals: A {\it terminal } is a literal string that appears in the language in hand.
		 \item Non-terminal: A {\it non-terminal } consists of Non-terminals and Terminals, used for structuring the language.
	\end{itemize}
	
	\subsection*{Production Rule}
		A production rule has the form: \\
		$N ::= X$, \\
		where $X$ is a Non-terminal or a Terminal. \\
		
		The elements in a production rule can be identified by: \\
		Terminals are written in {\bf bold } and 
		Non-terminals are just written in plain text.
	\subsection*{Grouping of production rules}
		If we have two production rules on the form:\\
		$N ::= X$ \\
		$N ::= Y$ \\
		we are allowed to make a transformation: \\
		$N ::= X | Y$ \\
		Meaning $N$ is $X$ or $Y$ \\
		
	\begin{comment}
	\subsection*{Example of a small BNF}
		\label{ex-bnf}
		\begin{center}
			\begin{tabular}{l l l}
				A		&	::=		&	 AB$ \mid $ B \\
				B		&	::=		&	{\bf b} $\mid$ {\bf n } $\mid$ {\bf f } \\
			\end{tabular}
		\end{center}
		This BNF would allow us to write programs like \\(We use $\rightarrow$ to show applying of a production rule): \\
		{\bf bnf}: A $\rightarrow$ AB $\rightarrow$ A{\bf f} $\rightarrow$ 
		AB{\bf f} $\rightarrow$A{\bf nf} $\rightarrow$ B{\bf nf}$\rightarrow${\bf bnf } \\
	
	
	
	\subsubsection{Terminals}
		\begin{longtable}[l]{l}
			$\{$\\
			$\}$\\
			$>$\\
			$<$\\
			$($\\
			$)$\\
			$=$\\
			$+$\\
			$-$\\
			$*$\\
			$/$\\
			$\&$\\
			$|$\\
			$"$\\
			$,$\\
			$.$\\
			$;$\\
			$Attack$\\
			$AttackSpeed$\\
			$Behaviour$\\
			$Config$\\
			$Damage$\\
			$Distance$\\
			$else$\\
			$Grid$\\
			$Health$\
			$Height$\\
			$if$\\
			$Maxima$\\
			$Melee$\\
			$MoveAway$\\
			$Movement$\\
			$MoveTowards$\\
			$Position$\\
			$Range$\\
			$Ranged$\\
			$Regiment$\\
			$RegimentPosition$\\
			$Regiments$\\
			$SearchForEnemies$\\
			$SearchForFriends$\\
			$Size$\\
			$Standards$\\
			$Team$\\
			$Teams$\\
			$Type$\\
			$While$\\
			$Width$\\
		\end{longtable}
%	\subsubsection{Nonterminals}
%		\begin{tabular}{l c}
%			Behaviour-Block   & Control-Structure   \\
%			Single-Command    & UnitStat 			\\
%			UnitStat-Name	  & Integer-Literal		\\
%			Comment			  & Team-File 			\\
%			Regiment-Block	  & Expression			\\
%			Operator		  & Config-File			\\
%			Grid-Block	      & Rules-Block 		\\
%			Maxima-Block	  & MaximaStat		\\
%			MaximaStat-Name & Standards-Block	 	\\
%			StandardsStat	  & StandardsStat-Name 	\\
%			Identifier		  & 					\\
%		\end{tabular}
	\end{comment}
	\subsubsection{BNF}

	\subsection{EBNF}
		{\it EBNF} stands for {\it Extendend Backus Naur Form} and is, as the name refers to an extension of BNF.
		The EBNF allows us to use regular expressions to express production rules, which makes our rule set more compact and 
		easier to implement.
	
	\subsubsection*{Regular expressions}
		A regular expression is a convenient way of expressing strings. We use the following regular expressions: \\
		\begin{itemize}
			\item *(star): States that the terminal or non-terminal must be used 0 or more times.
			\item Parentheses: Used for grouping of non-terminals and terminals.
			\item $\epsilon$: Represents the empty string.
		\end{itemize}
		Please note that these regular expressions can only be used on the right side of the production rules. \\
		
		By using these regular expression we can \\
		transform the BNF example given in \ref{ex-bnf} to an EBNF: \\
		
		\begin{tabular}{l l l}
			B		&	::=		&	({\bf b} $\mid$ {\bf n } $\mid$ {\bf f })* \\
		\end{tabular}
		
		It is easy to see that the transformation made the rules more compact (we removed one the the production rules), but
		how do we transform a BNF to a EBNF in a systematic way? 
		This can be done by applying {\it left factorization }, {\it elimination of left recursion} and {\it substitution of non-terminals }.
		
		\subsubsection*{Left factorization}
			If we have a production rule on the form: \\
			\begin{tabular}{l l l}
				T		&	::=		&	AB$\mid$AC \\
			\end{tabular}
			
			We can do a left factorization: \\
			\begin{tabular}{l l l}
				T		&	::=		&	A(B$\mid$C) \\
			\end{tabular}
			
			This can be done because T will always start with an A and end with either a B or C.
			
		\subsubsection*{Elimination of left recursion}
			If we have a production rule on the form: \\
			\begin{tabular}{l l l}
				T		&	::=		&	A$\mid$TB \\
			\end{tabular}
			
			We can do an elimination of left recursion: \\
			\begin{tabular}{l l l}
				T		&	::=		&	A(B)* \\
			\end{tabular}
			
			To understand how we can do this look at the first production rule. To terminate the production rule we need 
			to put an A, so T will always consist of an A. When we are not selecting an A(, and terminating) we are only making B's, which is
			the same as B*.
		\subsubsection*{Substitution of non-terminals}
			If we have two production rules on the form: \\
			
			\begin{tabular}{l l l}
				T		&	::=		&	U $\mid$ AUC \\
				U		&	::=		&	{\bf ab} $\mid$ {\bf ba} \\
			\end{tabular} \\
			
			We can do a substitution of the non-terminal U for every production rule. This means that we 
			substitute any occurrence of U in a production rule with what is on the right hand side of U like this: \\
			
			\begin{tabular}{l l l}
				T		&	::=		&	({\bf ab} $\mid$ {\bf ba}) $\mid$ A ({\bf ab} $\mid$ {\bf ba})B \\
			\end{tabular}
	\subsection{The EBNF of the WAR Language}
				     