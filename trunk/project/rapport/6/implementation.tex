\section{Simulator}
	This section will explain how the simulator is implemented, the diagram below \ref{fig:sim} shows the process.
\begin{figure}[H]
\center
\begin{tikzpicture}[->,>=stealth',shorten >=1pt,auto,node distance=2.8cm,
                    semithick]
	\tikzstyle{every state}=[draw,text=black,rectangle]
	\node[state] (A)								{GameDataRetriever};
	\node[state] (F)[left of=A,node distance=5cm]						{Checker};
	\node[state] (B)[below of=A] 					{GameDataValidator};
	\node[state] (C)[right of=B,node distance=8cm]	{Simulation Failed};
	\node[state] (D)[below of=B]					{GameState};
	\node[state,accepting] (E)[below of=D] 			{Simulation Success};
	
	\path 	(F) edge					node{DAST}		(A)
			(A) edge					node{Game Data} (B)
			(B)	edge					node{Validated Game Data} (D)
			(B) edge 					node{Error found} (C)
			(D) edge [loop right] 		node{Updating the current GameState}(D)
			(D) edge 					node{A team won} (E);
			
\end{tikzpicture}
\label{fig:sim}
\caption{Diagram showing the simulation}
\end{figure}

The following subsections will explain what happens at the phases, but first an explanation of how the game world is emulated.
\subsection{Game Classes}
	To emulate the game we have to create objects, which represent what is seen in the game such as regiments, the grid and the teams.
	This was simply done by making a class called {\it Regiment}, {\it Grid}, {\it Team} and {\it Tile}. 
	Team contains a list of regiments, Regiment have methods for the three {\it UnitFunctions}: {\it Attack}, 
	{\it MoveTowards} and {\it MoveAway} (see section \ref{rules:action}), Tile has a boolean to indicate that it is occupied 
	and \textit{Grid} contains a 2-dimensional array of {\it Tile}s. We call the game environment at a specific point in the simulation for a {\it GameState}. 

\subsection{GameDataRetriever}
	The job of the {\it GameDataRetriever} is to retrieve the game data from the {\it DASTs}. The class implements the visitor pattern
	(see section \ref{impl:visitorinterface}) to traverse the decorated abstract syntax tree and retrieve the data. 
	It starts by visiting each of the {\it Team file}s {\it DASTs} and when it visits the {\it Regiment node} it instantiates a new Regiment object, 
	with the {\it UnitStat}s is contains, after this it is added to a Team object. When game data for every team have been retrieved the {\it Config file}
	{\it DAST} is visited. The data of the {\it Standards block node} is simply saved as a Regiment Object, because it is so similar to Regiment retrieving. 
	The {\it Grid} data is retrieved from the Grid block. The {\it Maxima stats} is saved to a Maxima object, which contains a field for any of available 
	Maxima stats. After all data is retrieved it is passed to the {\it GameDataRetriever}

\subsection{GameDataValidator}
	After all the data have been retrieved from the DAST we have to validate if the data is correct. 
	By correct we mean that every regiment have all the {\it UnitStats} or that the {\it UnitStat} 
	of a regiment does not exceed the {\it MaximaStat} written. 
	The class {\it GameDataValidator} is started by using the Team objects, standards regiment and maxima object passed from the {\it GameDataRetriever}.
	The process of the validation: \\
	\begin{enumerate}
		\item Check if there are more teams than allowed according to the config file
		\item Check if the teams have more regiments than allowed according to the config file
		\item Check if every regiment does not exceed Maxima Stats
		\item Check if every regiment is missing UnitStats, if so the UnitStat from the standards regiment is given
		\item Check if any of the regiments have the same positions
	\end{enumerate}
	If any of the steps fail, an error is given and the simulation will not start. If it succeeds it returns a new {\it GameState} containing the teams 
	and the grid.
	
\subsection{Simulator}
	The simulator class starts by using the {\it GameDataRetriever} and {\it GameDataValidator} classes. The \textit{GameDataValidator} returns the first 
	\textit{GameState} if it succeeded. The simulator reads and modifies a \textit{GameState} in every round of the game. This is done by iterating through all 
	regiments from the teams and interpreting their \textit{Behaviour DAST}. This main loop is showed here: \\
	\begin{lstlisting}[basicstyle=\small\sffamily,
		keywords={break,case,const,continue,default,else,enum,
		for,if,return,switch,while,do,long,void,int,float,double,
		char,struct,typedef,include,size\_t},
		keywordstyle={\color{blue}},
		comment={[l]{//}}, morecomment={[s]{/*}{*/}}, commentstyle=\itshape,
		columns={[l]flexible}, numbers=left, numberstyle=\tiny,
		frameround=fftt, frame=shadowbox, captionpos=b,
		caption={Main loop of the simulation},
		label=impl:regimentloop]
foreach (Regiment regiment in regimentTurnOrder)
{
	if (regiment.currentSize > 0)
	{
		currentGameState = 
behaviourInterpreter.InterpreteBehaviour(regiment, currentGameState);
	}
	else
	{
	currentGameState.teams[regiment.team].regiments.Remove(regiment);
		if (currentGameState.teams[regiment.team].regiments.Count <= 0)
		{
			teamsLeft--;
		}
	}
}
	\end{lstlisting}
	{\it behaviourInterpreter} is an object of the class {\it BehaviourInterpreter}, which is able to interpret \textit{Behaviour DAST}s.
	
\subsection{BehaviourInterpreter}
	This class is the third and last class that implements the \textit{Visitor interface} (see section \ref{impl:visitorinterface}), which it uses to interpret the 
	DAST, this type of interpretation is called {\it Recursive Descent Interpretation}. The interpretation starts with the call of the
	method called {\it InterpreteBehaviour}, which can be seen below in \ref{impl:intbe}.
	\begin{lstlisting}[basicstyle=\small\sffamily,
		keywords={break,case,const,continue,default,else,enum,
		for,if,return,switch,while,do,long,void,int,float,double,
		char,struct,typedef,include,size\_t},
		keywordstyle={\color{blue}},
		comment={[l]{//}}, morecomment={[s]{/*}{*/}}, commentstyle=\itshape,
		columns={[l]flexible}, numbers=left, numberstyle=\tiny,
		frameround=fftt, frame=shadowbox, captionpos=b,
		caption={The method InterpreteBehaviour},
		label=impl:intbe]
public GameState InterpreteBehaviour(Regiment regiment, GameState gameState)
{
	regimentAssignments.Clear();
	currentGameState = gameState;
	currentRegiment = regiment;
	currentRegiment.currentMovement = currentRegiment.movement;
	currentRegiment.hasAttacked = false;
	currentRegiment.behaviour.Visit(this, null);
	return currentGameState;
}		
	\end{lstlisting}
	The methods starts with resetting the list regimentAssignments, which is a list of identifiers that have been assigned to regiments. To retrieve 
	a regiment from this list the method GetRegiment can be used, which takes a string as an argument.
	After this the method resets the current regiments movement and ability to attack and visits the {\it Behaviour DAST} of the regiment. 
	When class visits a node which needs interpretation, e.g. the {\it UnitFunction node} they will be interpreted. To show 
	how this is actually done we will now show interpretation of conditional statements, {\it UnitFunctions} and {\it RegimentSearch}.
	
	\subsubsection{Conditional Statements}
		Conditional statements consists of code that has to be executed and a condition, in our DAST the code that needs execution is a 
		{\it SingleCommand node} and the condition is an {\it Expression node}. 
		The code example \ref{impl:ifcmd} shows how a visit of the {\it IfCommand node} looks like.
		\begin{lstlisting}[basicstyle=\small\sffamily,
			keywords={break,case,const,continue,default,else,enum,
			for,if,return,switch,while,do,long,void,int,float,double,
			char,struct,typedef,include,size\_t},
			keywordstyle={\color{blue}},
			comment={[l]{//}}, morecomment={[s]{/*}{*/}}, commentstyle=\itshape,
			columns={[l]flexible}, numbers=left, numberstyle=\tiny,
			frameround=fftt, frame=shadowbox, captionpos=b,
			caption={VisitIfCommand method from the class BehaviourInterpreter},
			label=impl:ifcmd]
public Object VisitIfCommand(IfCommand ast, Object obj)
{
	//Retrieves the boolean value of the expression
	BoolValue b = (BoolValue)ast.e.Visit(this, null);
	
	//Checks if the expression evaluates to true
	if (b.b)
	{
		ast.sc1.Visit(this, null);
	}
	//Checks if there is any else if commands and visits them if there is
	else if (ast.eifc.Count > 0) { ast.eifc.ForEach(x => x.Visit(this, null)); }
	//Checks if there is any else command and visits it if there is
	else if (ast.sc2 != null) { ast.sc2.Visit(this, null); }
	return null;
}	
		\end{lstlisting}
		The evaluation of the condition is done by visiting the expression node, which is tied to the current conditional statement. 
		The simple expressions (IntegerExpression or RegimentStatExpression) will simply return their value, but the {\it BinaryExpression} 
		has to evaluate what the expression is by using a method called {\it CheckBinaryExpression}, which returns the value of the expression.
		\begin{lstlisting}[basicstyle=\small\sffamily,
			keywords={break,case,const,continue,default,else,enum,
			for,if,return,switch,while,do,long,void,int,float,double,
			char,struct,typedef,include,size\_t},
			keywordstyle={\color{blue}},
			comment={[l]{//}}, morecomment={[s]{/*}{*/}}, commentstyle=\itshape,
			columns={[l]flexible}, numbers=left, numberstyle=\tiny,
			frameround=fftt, frame=shadowbox, captionpos=b,
			caption={Code snippet from CheckBinaryExpression},
			label=impl:chkbexp]
if (v1 is IntValue)
{
	int i1 = ((IntValue)v1).i;
	int i2 = ((IntValue)v2).i;
	BoolValue b = new BoolValue();
	IntValue i = new IntValue();
	switch (op)
	{
	case "<": if (i1 < i2) { b.b = true; } else { b.b = false; } return b;
	case ">": if (i1 > i2) { b.b = true; } else { b.b = false; } return b;
	case "==": if (i1 == i2) { b.b = true; } else { b.b = false; } return b;
	case ">=": if (i1 >= i2) { b.b = true; } else { b.b = false; } return b;
	case "<=": if (i1 <= i2) { b.b = true; } else { b.b = false; } return b;
	case "!=": if (i1 != i2) { b.b = true; } else { b.b = false; } return b;

	case "+": i.i = i1 + i2; return i;
	case "-": i.i = i1 - i2; return i;
	case "/": i.i = i1 / i2; return i;
	case "*": i.i = i1 * i2; return i;
	}
}
		\end{lstlisting}
		The code snippet \ref{impl:chkbexp} shows how the method handles binary expressions 
		where both the expressions are integers, here {\it op} is a string which represent the operator of the binary expression. 
		CheckBinaryExpression has similar handling for expressions with type expressions.
		
	\subsubsection{UnitFunctions}
		UnitFunctions are the three actions (see section \ref{rules:action}) that a regiment can perform.
		\begin{lstlisting}[basicstyle=\small\sffamily,
			keywords={break,case,const,continue,default,else,enum,
			for,if,return,switch,while,do,long,void,int,float,double,
			char,struct,typedef,include,size\_t},
			keywordstyle={\color{blue}},
			comment={[l]{//}}, morecomment={[s]{/*}{*/}}, commentstyle=\itshape,
			columns={[l]flexible}, numbers=left, numberstyle=\tiny,
			frameround=fftt, frame=shadowbox, captionpos=b,
			caption={VisitUnitFunction from the class },
			label=impl:visituf]	
public Object VisitUnitFunction(UnitFunction ast, Object obj)
{
	//spelling of the regiment identifier
	string spelling = (string)ast.i.Visit(this, null);
	
	//spelling of the function name
	string functionName = (string)ast.ufn.Visit(this, null);
	
	//Gets the regiment by using the method GetRegiment
	Regiment regiment = GetRegiment(spelling);
	if (regiment != null)
	{
//Checks which UnitFunction we are trying to call and calls it from the regiment class
		switch (functionName)
		{
		case "Attack": currentRegiment.Attack(regiment); break;
		case "MoveTowards": currentRegiment.MoveTowards(regiment); break;
		case "MoveAway": currentRegiment.MoveAway(regiment); break;
		}
	}
	return null;
}		\end{lstlisting}
		The code snippet \ref{impl:visituf} shows how a visit of the {\it UnitFunction node} looks like. 
		First the identifier representing the regiment is retrieved and then the UnitFunction type is retrieved.
		The identifier, which was retrieved is then used in the function GetRegiment that returns a regiment that matches the identifier.
		Using a switch case the correct action of the current regiment is performed.
		
	\subsubsection{RegimentSearch}
		The {\it RegimentSearch methods} is used for finding either enemy or friendly regiments on 
		the grid (see usage at section \ref{sec:regimentsearch}). 
		We will here explain how the the visit to a {\it RegimentSearch node} looks like in multiple code snippets.
		\begin{lstlisting}[basicstyle=\small\sffamily,
			keywords={break,case,const,continue,default,else,enum,
			for,if,return,switch,while,do,long,void,int,float,double,
			char,struct,typedef,include,size\_t},
			keywordstyle={\color{blue}},
			comment={[l]{//}}, morecomment={[s]{/*}{*/}}, commentstyle=\itshape,
			columns={[l]flexible}, numbers=left, numberstyle=\tiny,
			frameround=fftt, frame=shadowbox, captionpos=b,
			caption={1. part of visit of RegimentSearch node in the class BehaviourInterpreter},
			label=impl:regsearch1]		
public Object VisitRegimentSearch(RegimentSearch ast, Object obj)
{
	//Retrieves the type of regimentsearch we are doing
	string regimentSearchSpelling = (string)ast.rsn.Visit(this, null);

	//A list for storing all the regiments we find
	List<Regiment> regimentsFound = new List<Regiment>();
	if (regimentSearchSpelling == "SearchForFriends")
	{
		//Looks for regiments which are friends
	regimentsFound = currentGameState.teams[currentRegiment.team].regiments;
	}
	else
	{
		//Finds all the regiments which is not friendly
		foreach (Team team in currentGameState.teams)
		{
			if (team.number != currentRegiment.team)
			{
				regimentsFound.AddRange(team.regiments);
			}
		}
	}
		\end{lstlisting}
		The first part of the method tries to find out which regiments the search is for where {\it regimentsFound} holds this collection. 
		If we are searching for friends then only the regiments matching the team of the current regiment will be added to {\it regimentsFound}.
		Else we add all the regiments, which is not matching the current regiments team.
		\begin{lstlisting}[basicstyle=\small\sffamily,
			keywords={break,case,const,continue,default,else,enum,
			for,if,return,switch,while,do,long,void,int,float,double,
			char,struct,typedef,include,size\_t},
			keywordstyle={\color{blue}},
			comment={[l]{//}}, morecomment={[s]{/*}{*/}}, commentstyle=\itshape,
			columns={[l]flexible}, numbers=left, numberstyle=\tiny,
			frameround=fftt, frame=shadowbox, captionpos=b,
			caption={1. part of visit of RegimentSearch node in the class BehaviourInterpreter},
			label=impl:regsearch2]
	if (ast.p != null)
	{
		foreach (Parameter p in ast.p)
		{
			//Retrieves the parameter from the ast
			Parameter parameter = (Parameter)p.Visit(this, null);

		//Gets all the regiments which matches the parameter
		regimentsFound = GetRegiments(regimentsFound, parameter);
			if (regimentsFound.Count == 0) { break; }
		}
	}
	if (regimentsFound.Count == 0)
	{
		return null;
	}
	return GetClosestRegiment(regimentsFound);
}
	\end{lstlisting}
	This part of the method is about applying the parameters passed to the {\it RegimentSearch function} on the list of regiments we just found. 
	The parameters are applied one of a time using the method GetRegiments, which returns the regiments for which the parameters are true. 
	If no regiments was found that fits the parameters the method returns null, otherwise we get the regiment which is closest to the current regiment 
	using the method GetClosestRegiment. 