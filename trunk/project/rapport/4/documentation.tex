This chapter will show how to program scripts for the simulator. Scripts written for the simulator is written in the {\it WAR language}. 
When a term is defined we use $<$Term$>$ to refer to the given term. 
When a multiple terms are used we illustrate it like this: \\

\begin{lstlisting}[basicstyle=\small\sffamily,
keywordstyle={\color{blue}},
comment={[l]{//}}, morecomment={[s]{/*}{*/}}, commentstyle=\itshape,
columns={[l]flexible}, numbers=left, numberstyle=\tiny,
frameround=fftt, frame=shadowbox, captionpos=b,
caption={Scripting}]
<Term-1>
..
.
<Term-n>
\end{lstlisting}
This illustrates that n terms can be written here. \\

To run a simulation you need a {\it Config File} (See section \ref{doc:cfgfile}) with the file ending {\it .cfg} and at least two {\it Team File}s 
(See section \ref{doc:teamfile}) with the file ending {\it .war}.

\section{Identifier}
	An {\it Identifier} is a word which starts with a letter and then follows a sequence of letters and/or digits.
	Identifiers is used for names for {\it Blocks} (See section \ref{doc:blocks}) and for {\it Regiment Assignment} (See section \ref{doc:regass}).
\section{Reserved Keywords}

The following table \ref{reserved_keywords} shows the reserved words of the WAR language. One cannot use the reserved words as an identifier.

\begin{table}[H]

\begin{tabular}{|c | c | c|}
\hline
 Team			&			Regiment					&		Rules				\\
\hline
 Maxima		&			Config						&		Behaviour 		\\						
\hline
 if					&			else							&		 while 				\\							
\hline
 Attack			&			SearchForEnemies 	&	SearchForFriends	\\					
\hline
 Distance		&		 	Size							&		Type					\\ 									
\hline
 Melee			&			Movement				&		Position			\\								
\hline
 Ranged		&			Range						&		Damage			\\								
\hline
 Health			&			Movement				&		AttackSpeed		\\						
\hline
 RegimentPosition & 								&								\\				
\hline
\end{tabular}
\caption{Reserved keywords}
\label{reserved_keywords}
\end{table}
\section{Data Types}
	In WAR we use the following data types: \\
	\begin{itemize}
		\item Integer
		\item Boolean
		\item Position
		\item Regiment
		\item AttackType
	\end{itemize}
	To show the need for a element of a given data type we use $<$DataType$>$.
	To show that an identifier is of a specific data type we write $<$Identifier:DataType$>$.

\section{Unitstats}
	Unit stats defines the stats of a regiment. The unitstats are used in both the config and team file. Here are the unitstats used in WAR
		\paragraph{Size}
			Sets the size of the regiment, it's an integer.
			If the size of the regiment is bigger than 1 it will cover multiple tiles.

\begin{lstlisting}[basicstyle=\small\sffamily,
keywordstyle={\color{blue}},
comment={[l]{//}}, morecomment={[s]{/*}{*/}}, commentstyle=\itshape,
columns={[l]flexible}, numbers=left, numberstyle=\tiny,
frameround=fftt, frame=shadowbox, captionpos=b,
caption={Size of a regiment}]
Size = <Int>;
\end{lstlisting}
		\paragraph{Position}
			Sets the position of the regiment. If the size of the regiment is bigger than 1 them 
			only 1 tile of the regiment touches the position. Usage: 

\begin{lstlisting}[basicstyle=\small\sffamily,
keywordstyle={\color{blue}},
comment={[l]{//}}, morecomment={[s]{/*}{*/}}, commentstyle=\itshape,
columns={[l]flexible}, numbers=left, numberstyle=\tiny,
frameround=fftt, frame=shadowbox, captionpos=b,
caption={Position of the regiment}]
RegimentPosition = Position(<Int>,<Int>);
\end{lstlisting}
		\paragraph{Range} sets how many tiles away the regiment can attack another regiment. 
		
\begin{lstlisting}[basicstyle=\small\sffamily,
keywordstyle={\color{blue}},
comment={[l]{//}}, morecomment={[s]{/*}{*/}}, commentstyle=\itshape,
columns={[l]flexible}, numbers=left, numberstyle=\tiny,
frameround=fftt, frame=shadowbox, captionpos=b,
caption={Range}]
Range = <Int>;
\end{lstlisting}
		\paragraph{Type}
			Type defines which way the regiment attacks. The type can either be Melee or Ranged.
			Melee means that the regiment can only attack other regiments which it touches. Ranged means 
			that the regiment can attack any regiment where the distance in number of tiles is less than the Range.

		\begin{lstlisting}[basicstyle=\small\sffamily,
		keywordstyle={\color{blue}},
		comment={[l]{//}}, morecomment={[s]{/*}{*/}}, commentstyle=\itshape,
		columns={[l]flexible}, numbers=left, numberstyle=\tiny,
		frameround=fftt, frame=shadowbox, captionpos=b,
		caption={Type of regiment}]
Type = <AttackType>; 
			\end{lstlisting}	
		\paragraph{Damage} defines how much Health an opposing regiment lose when attacked.

		\begin{lstlisting}[basicstyle=\small\sffamily,
		keywordstyle={\color{blue}},
		comment={[l]{//}}, morecomment={[s]{/*}{*/}}, commentstyle=\itshape,
		columns={[l]flexible}, numbers=left, numberstyle=\tiny,
		frameround=fftt, frame=shadowbox, captionpos=b,
		caption={Damage of the regiment}]
Damage = <Int>;
			\end{lstlisting}
		\paragraph{Movement} means how many tiles the regiment can move.

		\begin{lstlisting}[basicstyle=\small\sffamily,
		keywordstyle={\color{blue}},
		comment={[l]{//}}, morecomment={[s]{/*}{*/}}, commentstyle=\itshape,
		columns={[l]flexible}, numbers=left, numberstyle=\tiny,
		frameround=fftt, frame=shadowbox, captionpos=b,
		caption={Movement of the regiment}]
Movement = <Int>;
			\end{lstlisting}
		\subparagraph{AttackSpeed} defines how many attacks a regiment can attack in a turn.

		\begin{lstlisting}[basicstyle=\small\sffamily,
		keywordstyle={\color{blue}},
		comment={[l]{//}}, morecomment={[s]{/*}{*/}}, commentstyle=\itshape,
		columns={[l]flexible}, numbers=left, numberstyle=\tiny,
		frameround=fftt, frame=shadowbox, captionpos=b,
		caption={AttackSpeed of the regiment}]
AttackSpeed = <Int>;
			\end{lstlisting}
			
\paragraph{An example} of the above UnitStats.
		\begin{lstlisting}[basicstyle=\small\sffamily,
		keywordstyle={\color{blue}},
		comment={[l]{//}}, morecomment={[s]{/*}{*/}}, commentstyle=\itshape,
		columns={[l]flexible}, numbers=left, numberstyle=\tiny,
		frameround=fftt, frame=shadowbox, captionpos=b,
		caption={Example: Using the UnitStats}]
Regiment PirateWarrior
{
	Size = 10;
	Type = Melee;
	Damage = 8;
	Health = 4;
	Movement = 1;
	AttackSpeed = 1;
	RegimentPosition = Position(2,3);
	...
}
\end{lstlisting}
			

\section{UnitStatType}
	A {\it UnitStatType} is any of the UnitStats, except for Position, which is replaced by {\it Distance}. We use the UnitStatTypes in 
	{\it Regiment Assignment} and {\it RegimentStat}.
	Distance is defined as:
	\paragraph{Distance} is an integer, which expresses how many tiles a given regiment is away.

\section{Regiment Assignment}
\label{doc:regass}
	One is able to assign an identifier to a given regiment, that way one can use the identifier when attacking or moving.
	
		\begin{lstlisting}[basicstyle=\small\sffamily, keywordstyle={\color{blue}}, comment={[l]{//}}, morecomment={[s]{/*}{*/}}, commentstyle=\itshape, columns={[l]flexible}, numbers=left, numberstyle=\tiny, frameround=fftt, frame=shadowbox, captionpos=b,
		caption={Regiment Assignment}]
Regiment <Identifier> = <RegimentSearchFunction>			
		\end{lstlisting}	

		\begin{lstlisting}[basicstyle=\small\sffamily, keywordstyle={\color{blue}}, comment={[l]{//}}, morecomment={[s]{/*}{*/}}, commentstyle=\itshape, columns={[l]flexible}, numbers=left, numberstyle=\tiny, frameround=fftt, frame=shadowbox, captionpos=b,
		caption={Example of a regiment assignment}]
Regiment PirateWarrior
{
... //Here comes UnitStats
}
		\end{lstlisting}
		

\section{RegimentStat}
	A {\it RegimentStat} is used to get the stats of a regiment. A RegimentStat can be used in Expressions. Usage:\\

		\begin{lstlisting}[basicstyle=\small\sffamily,
		keywordstyle={\color{blue}},
		comment={[l]{//}}, morecomment={[s]{/*}{*/}}, commentstyle=\itshape,
		columns={[l]flexible}, numbers=left, numberstyle=\tiny,
		frameround=fftt, frame=shadowbox, captionpos=b,
		caption={Regiment Stat}]
<Identifier:Regiment>.<UnitStatType>
	\end{lstlisting}
	Please note that the identifier must be from a Regiment Assignment.

		\begin{lstlisting}[basicstyle=\small\sffamily,
		keywordstyle={\color{blue}},
		comment={[l]{//}}, morecomment={[s]{/*}{*/}}, commentstyle=\itshape,
		columns={[l]flexible}, numbers=left, numberstyle=\tiny,
		frameround=fftt, frame=shadowbox, captionpos=b,
		caption={Example of an assignment in use}
		label={example:regi_assignment}]
Regiment PirateArcher
{
	Size = 10;
	Type = Ranged;
	Range = 2;
	Damage = 4;
	Health = 4;
	Movement = 1;
	AttackSpeed = 1;
	RegimentPosition = Position(3,2);
	Behaviour SilentMonkeysBehavoiur1
	{
		Regiment enemy = SearchForEnemies();
		if (enemy.Distance <= Range)
		{
			Attack(enemy);
			if(enemy.Distance < 2)
			{
				MoveAway(enemy);
			}
		}
		else
		{
			MoveTowards(enemy);
		}
	}
}
\end{lstlisting}
In listing \ref{example:regi_assignment} line 5 we see that the range is getting assigned the value of 2, and later in line 13 the range is getting used in an \textit{if} statement, whether the enemy is in range the PirateArcher will attack.

\section{Blocks}
\label{doc:blocks}
	In WAR there are 5 different types of blocks: Regiment, Grid, Standards, Behaviour and Maxima blocks. 
	These blocks are being used in both Config (See \ref{doc:teamfile}) and Team file (See \ref{doc:teamfile}).
	\subsection{Grid block}
		A grid block contains the {\it Height} and the {\it Width }of the map.

		\begin{lstlisting}[basicstyle=\small\sffamily,
		keywordstyle={\color{blue}},
		comment={[l]{//}}, morecomment={[s]{/*}{*/}}, commentstyle=\itshape,
		columns={[l]flexible}, numbers=left, numberstyle=\tiny,
		frameround=fftt, frame=shadowbox, captionpos=b,
		caption={Grid Block}]
Grid <Identifier>
{
	Width = <Int>;
	Height = <Int>;
}
		\end{lstlisting}	
		The Width determines how many tiles wide the map will be and Height determines how many tiles high the map will be.
	\paragraph{Behaviour Block}
		The behaviour block determines how a regiment will behave in the simulation. To control the a regiment one can 
		use the conditional statements {\it if-else if-else} and a {\it while loop} and using unit functions. Please note that 
		conditional statements and expressions can only be used in behaviour blocks.
		\subparagraph{if-else if-else Statement} Example of the structure of an {\it if-else if-else} statement:

		\begin{lstlisting}[basicstyle=\small\sffamily,
		keywordstyle={\color{blue}},
		comment={[l]{//}}, morecomment={[s]{/*}{*/}}, commentstyle=\itshape,
		columns={[l]flexible}, numbers=left, numberstyle=\tiny,
		frameround=fftt, frame=shadowbox, captionpos=b,
		caption={if and else if statements}]
if(<Expression>)
{
	//Code
}
else if (<Expression>)
{
	//Code
}
	else
{
	//Code
}
			\end{lstlisting}
			If the expression inside the parenthesis after if or else if is evaluated to true then the code inside the block will be executed.
			If none of the expressions is evaluated to true the the code inside the else block will be executed. Please not that only the if block
			is required, there can be 1 to multiple else if blocks and only 1 else block which has to end the statement. \\
		\subparagraph{while Loop}
			Structure of a {\it while} loop: \\			

		\begin{lstlisting}[basicstyle=\small\sffamily,
		keywordstyle={\color{blue}},
		comment={[l]{//}}, morecomment={[s]{/*}{*/}}, commentstyle=\itshape,
		columns={[l]flexible}, numbers=left, numberstyle=\tiny,
		frameround=fftt, frame=shadowbox, captionpos=b,
		caption={While loop}]
while(<Expression>)
{
//Code
}
			\end{lstlisting}
			If the expression is evaluated to true the code inside the block is executed. After it is executed the expression will be checked again.
		\subparagraph{Unit Functions}
			Unit functions are used to move the regiment or attacking with the regiment.

		\begin{lstlisting}[basicstyle=\small\sffamily,
		keywordstyle={\color{blue}},
		comment={[l]{//}}, morecomment={[s]{/*}{*/}}, commentstyle=\itshape,
		columns={[l]flexible}, numbers=left, numberstyle=\tiny,
		frameround=fftt, frame=shadowbox, captionpos=b,
		caption={Unit functions}]
MoveTowards(<Regiment>);
MoveAway(<Regiment>);
Attack(<Regiment>);
			\end{lstlisting}
	\paragraph{Standards block}
		A standards block consist of unit stats and a behaviour block. 

		\begin{lstlisting}[basicstyle=\small\sffamily,
		keywordstyle={\color{blue}},
		comment={[l]{//}}, morecomment={[s]{/*}{*/}}, commentstyle=\itshape,
		columns={[l]flexible}, numbers=left, numberstyle=\tiny,
		frameround=fftt, frame=shadowbox, captionpos=b,
		caption={Standards block}]
Standards
{
	<Unit-Stat-1>
	..
	.
	<Unit-Stat-6>
	<Behaviour-Block>
}
		\end{lstlisting}
		If a regiment in one of the team files haven't declared all the unit stats
		or defined a behaviour, then the stats/behaviour will be taken from the standards block.
	\subsection{Maxima Block}
		The Maxima block determines what are the limits for the unit stats. It also sets the maximum number of teams and regiments allowed.
		In the block one is allowed to write UnitStat and MaximaStat.
		MaximaStat is one of these:

		\begin{lstlisting}[basicstyle=\small\sffamily,
		keywordstyle={\color{blue}},
		comment={[l]{//}}, morecomment={[s]{/*}{*/}}, commentstyle=\itshape,
		columns={[l]flexible}, numbers=left, numberstyle=\tiny,
		frameround=fftt, frame=shadowbox, captionpos=b,
		caption={Maxima block}]
Teams = <Int>;
Regiments = <Int>;
		\end{lstlisting}
		The Maxima Block is written as:

		\begin{lstlisting}[basicstyle=\small\sffamily,
		keywordstyle={\color{blue}},
		comment={[l]{//}}, morecomment={[s]{/*}{*/}}, commentstyle=\itshape,
		columns={[l]flexible}, numbers=left, numberstyle=\tiny,
		frameround=fftt, frame=shadowbox, captionpos=b,
		caption={Maxima stat}]
Maxima
{
	<UnitStat-1> or <MaximaStat-1>
	..
	.
	<UnitStat-6> or <MaximaStat-2>
}
		\end{lstlisting}
	\subsection{Regiment Block}
		A Regiment block contains unit stats and a behaviour block.

		\begin{lstlisting}[basicstyle=\small\sffamily,
		keywordstyle={\color{blue}},
		comment={[l]{//}}, morecomment={[s]{/*}{*/}}, commentstyle=\itshape,
		columns={[l]flexible}, numbers=left, numberstyle=\tiny,
		frameround=fftt, frame=shadowbox, captionpos=b,
		caption={Regiment block}]
Regiment <Identifier>
{
<Unit-Stat-1>
..
.
<Unit-Stat-6>
<Behaviour-Block>
}
		\end{lstlisting}
\section{RegimentSearch Functions}
\label{sec:regimentSearch}
	To get information of regiments on the map one can use {\it RegimentSearch Functions}. There are two RegimentSearch Functions in WAR.\\

		\begin{lstlisting}[basicstyle=\small\sffamily,
		keywordstyle={\color{blue}},
		comment={[l]{//}}, morecomment={[s]{/*}{*/}}, commentstyle=\itshape,
		columns={[l]flexible}, numbers=left, numberstyle=\tiny,
		frameround=fftt, frame=shadowbox, captionpos=b,
		caption={RegimentSearch Functions}]
SearchForEnemies(Parameters);
SearchForFriend(Parameters);
	\end{lstlisting}
	Here {\it Parameters} is one or more expressions made with UnitStatType. Between two expressions there has to be a comma (,).
	Example: \\

		\begin{lstlisting}[basicstyle=\small\sffamily,
		keywordstyle={\color{blue}},
		comment={[l]{//}}, morecomment={[s]{/*}{*/}}, commentstyle=\itshape,
		columns={[l]flexible}, numbers=left, numberstyle=\tiny,
		frameround=fftt, frame=shadowbox, captionpos=b,
		caption={Functions with parameters}]
SearchForEnemies(Distance > 5,Size == 200);
	\end{lstlisting}
	This will return all the enemy regiments, which are more than 5 tiles away and is of Size 200.		
\section{Team File}
\label{doc:teamfile}
	A team file is written this way: \\

		\begin{lstlisting}[basicstyle=\small\sffamily,
		keywordstyle={\color{blue}},
		comment={[l]{//}}, morecomment={[s]{/*}{*/}}, commentstyle=\itshape,
		columns={[l]flexible}, numbers=left, numberstyle=\tiny,
		frameround=fftt, frame=shadowbox, captionpos=b,
		caption={Regiment Assignment},
		label=RegimentAssignment]
Team <Identifier>
<Regiment-Block-1>
...
..
.
<Regiment-Block-n>
		\end{lstlisting}
	
\section{Config File}
\label{doc:cfgfile}
	The config file sets the rules for the simulation, it can be limitations on the regiments and deciding the size of the map.
	A config file is written this way: \\

		\begin{lstlisting}[basicstyle=\small\sffamily,
		keywordstyle={\color{blue}},
		comment={[l]{//}}, morecomment={[s]{/*}{*/}}, commentstyle=\itshape,
		columns={[l]flexible}, numbers=left, numberstyle=\tiny,
		frameround=fftt, frame=shadowbox, captionpos=b,
		caption={Config file}]
Config
<Grid-Block>
<Standards-Block>
<Maxima-Block>
		\end{lstlisting}
	
