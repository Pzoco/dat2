\section{The Idea}
	We drew inspiration from many different angles, in order to come up with an interesting aspect that was feasible with our current technical abilities and the basic purpose of doing the project.
	The simulation is mainly inspired by things such as Tabletop gaming, Real-Time Strategy games, Chess, coming together in a rather simple package to provide with a somewhat interesting abstraction for a simulation which is implemented in our own programming language.
	The purpose of this programming language-to-be 
	is to create a scripting language for some sort of simulation. 
	During our process of consideration we thought out many different concepts, 
	and ultimately 
	\begin{comment}we came to a decision that it would be more interesting to 
	have a controlled environment with a lot of units participating. Reasons for 
	this is that the simulation would not be very interesting to perform when dealing with minor numbers, 
	the reasoning behind this is that we do not expect to make a very complicated simulation, therefore dealing in larger numbers proved 
	more interesting to us. \newline
	
	Ultimately\end{comment}
	we decided to do a war simulation, with two or more opposing forces 
	spawning in a grid like fashion. These forces would be composed of regiments of 
	different types, with each regiment having a captain, or a hero. The reasoning for having a 
	captain is to introduce an element which could lead to some interesting event handling, 
	eg. if the captain dies, something happens to the regiment, such as a decline of morale, speed, or behaviour. \newline
	
	Interesting aspects of a simulation of this sort, is to establish basic ground rules, 
	and see how the simulation plays out on it's own, given basic instructions for how to start the simulation.