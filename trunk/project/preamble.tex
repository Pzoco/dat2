% Jeg forslår at man lige knytter en kommentar til de pakker man tilføjer så andre også kan benytte dem.

\usepackage[english]{babel} % Forklarer at dokumentet er engelsk
\usepackage{import}
\usepackage{verbatim} 
\usepackage[utf8]{inputenc} % Forklarer hvilket charset er dokumentet skrevet i
\usepackage[pdfborder={0 0 0 0}]{hyperref} % Indsætter url-adresser og ref-links korrekt. Disse kan vises som url adressen eller bare tekst og er klik-bare i pdf dokumentet. \href{URL}{text}
\usepackage{graphicx} % Bruges til at inkludere billede filer...
\usepackage{float} % Bruges blandt andet til H i figures
\bibliographystyle{plain} % Beskriver hvilken bibliografi stil der benyttes.
\usepackage[none]{hyphenat}
\usepackage{amsmath}
\usepackage{listings}
\usepackage{rotating} %Bruges til at vende teksten vertikalt
\usepackage{framed} % Benyttes til at lave bokse om diverse ting.
\usepackage{multirow} % Benyttes til avancerede tabeller
\usepackage{pdfpages} % Bruges til indsættelse af pdf sider i LaTeX dokumentet.
\usepackage{fixme}


\newtheorem{def:recommend}{Definition} % Benyttes til definitioner i studierapporten
