\documentclass{article}
\usepackage{a4wide}
\usepackage{listings,color}
\usepackage[lastexercise]{exercise}
\renewcommand{\ExerciseListHeader}{ 
\textbf{Exercise} \ExerciseHeaderNB:\quad} 
\renewcommand{\QuestionNB}{{\bf Q}:\quad}
\renewcommand{\AnswerListHeader}{ \textbf{A}:\quad} 
\pagestyle{empty}

\title{MVP Assignment 4 - MPI}
\date{Due date: 16/5/2011}
\begin{document}
\maketitle

% Leave this commented for answering.
 \newcommand{\question}[1]{#1}
 \newcommand{\answer}[1]{}

% Leave this uncommented for answering.
%\newcommand{\question}[1]{}
%\newcommand{\answer}[1]{{#1}}

%
% Write your group room and group number here.
%
\answer{
\begin{flushleft}
{\bf Group room:} FIXME\\
{\bf Group number:} FIXME
\end{flushleft}
}

\section{OpenMP}

The idea is to parallelize the matrix
multiplication with OpenMP in a very easy way.

\begin{ExerciseList}
\Exercise Copy \texttt{pmatrix.c} from your previous assignment.

\Question \emph{Parallelize the rearranged and the block-matrix
  multiplications using OpenMP.}

\end{ExerciseList}

% Your code goes here.
\begin{lstlisting}[basicstyle=\small\sffamily,
keywords={break,case,const,continue,default,else,enum,
for,if,return,switch,while,do,long,void,int,float,double,
char,struct,typedef,include,size\_t},
keywordstyle={\color{blue}},
comment={[l]{//}}, morecomment={[s]{/*}{*/}}, commentstyle=\itshape,
columns={[l]flexible}, numbers=left, numberstyle=\tiny,
frameround=fftt, frame=shadowbox, captionpos=b,
caption={Your OpenMP implementation.},
label=LST:openmp]

void mat_mult2(const double* a, const int* b, int* c, size_t n)
{
    int i, j, k;              /* You will need at least these. */
    assert(a != c && b != c); /* Check precondition. */

    /* Write here your matrix multiplication. */
	#pragma omp parallel shared(a, b, c, n)\
		private (i , j , k)
	{
	#pragma omp parallel for private (i)
    for(i = 0;i<n;i++)
		{
		#pragma omp parallel for private (k)
        for(k= 0;k<n;k++)
        {
            double temp = 0;
			#pragma omp parallel for private (j)
            for(j = 0;j<n;j++)
            {
                temp+= a[i*n+k]*b[k*n+j];
                c[i*n+j] +=temp;
			}
	    }
		}
	}
}


void mat_mult3(double *a, double *b, double *c, size_t n)
{
    size_t bi, bj, bk, i, j, k, maxi, maxj, maxk;

	#pragma omp parallel shared(a, b, c, n)\
		private(bi, bj, bk, i, j, k, maxi, maxj, maxk, temp)
		{
		#pragma omp parallel for private(bi)
    for(bi = 0;bi<n;bi+=BLOCK)
    {
	
        maxi = min(bi+BLOCK,maxi);
		#pragma omp parallel for private(bj)
        for(bj = 0;bj<n;bj+=BLOCK)
        {
            maxj = min(bj+BLOCK,maxj);
			#pragma omp parallel for private(bk)
            for(bk = 0;bk < n;bk+=BLOCK)
            {
                maxk = min(bk+BLOCK,maxk);
				#pragma omp parallel for private(i)
                for(i = 0;i<maxi;i++)
                {
					#pragma omp parallel for private(j)
                    for(j= 0;j<maxj;j++)
                    {
                        double temp = 0;
						#pragma omp parallel for private(k)
                        for(k = 0;k<maxk;k++)
                        {
                        temp+= a[i*n+k]*b[k*n+j];
                        }
                        c[i*n+j] +=temp;
                    }
                }
            }
        }
    }
	}
}
\end{lstlisting}

\section{MPI}

\begin{ExerciseList}
\Exercise Warm-up with a hello-world program.
\Question \emph{Complete the hello-world program so that every process
  prints its rank and the size of the (world) communicator.}
\end{ExerciseList}

% Your code goes here.
\begin{lstlisting}[basicstyle=\small\sffamily,
keywords={break,case,const,continue,default,else,enum,
for,if,return,switch,while,do,long,void,int,float,double,
char,struct,typedef,include,size\_t},
keywordstyle={\color{blue}},
comment={[l]{//}}, morecomment={[s]{/*}{*/}}, commentstyle=\itshape,
columns={[l]flexible}, numbers=left, numberstyle=\tiny,
frameround=fftt, frame=shadowbox, captionpos=b,
caption={Your hello-world implementation.},
label=LST:hello]
int main(int argc, char *argv[])
{

int rank, size;

MPI_Init(&argc, &argv); // Init
MPI_Comm_size(MPI_COMM_WORLD, &size); // Get the size
MPI_Comm_rank(MPI_COMM_WORLD, &rank); // Get the rank
printf("Process number %d of a total of %d says HELLO WORLD! \n", rank+1, size); // Hello World
MPI_Finalize();              // Finalize

return 0;

}

\end{lstlisting}

\begin{ExerciseList}
\Exercise \question{You will experiment blocking and non-blocking
  communications and deadlock issue. The goal is to have all processes
  starting with sending to the next process in the ring its own rank
  (times $n$). Then the processes forward the messages around $size$
  times.}
\answer{Blocking communication.}
  \Question \emph{Complete the cycle-mpi program with MPI\_Send.}

\Exercise
 \question{As it turns out, the OpenMPI manual states ``This  routine
  will  block until the message is sent to the destination.'' whereas
  the LAM manual says `` This  function  may  block until the message
  is received. Whether or not MPI\_Send blocks depends on factors such
  as how large the message is, how many messages are pending to the
  specific destination, whether LAMD or C2C communication is being
  used, etc.''.}
\answer{Implementation features.}
  \Question
  \emph{Experiment with $n$ to see when the sending
    becomes really blocking and find the deadlock.}

  \Question \emph{Fix the deadlock by breaking the cycle-dependency.}

  \Exercise
\question{It is possible to break the deadlock by using non-blocking
  communication instead.}
\answer{Non-blocking communication.}
  \Question \emph{Fix the deadlock by using the non-blocking MPI\_Isend.}

\end{ExerciseList}

% Your code goes here.
\begin{lstlisting}[basicstyle=\small\sffamily,
keywords={break,case,const,continue,default,else,enum,
for,if,return,switch,while,do,long,void,int,float,double,
char,struct,typedef,include,size\_t},
keywordstyle={\color{blue}},
comment={[l]{//}}, morecomment={[s]{/*}{*/}}, commentstyle=\itshape,
columns={[l]flexible}, numbers=left, numberstyle=\tiny,
frameround=fftt, frame=shadowbox, captionpos=b,
caption={Cycle 1: Blocking, deadlock.},
label=LST:cycle1]

void test(int n)
{
    int rank, size, i;
    MPI_Status status;
	MPI_Comm_rank(MPI_COMM_WORLD, &rank);
	MPI_Comm_size(MPI_COMM_WORLD, &size); 
    int *send_data = (int*) malloc(n*sizeof(int));
    int *recv_data = (int*) malloc(n*sizeof(int));

    if (!send_data || !recv_data)
    {
        fprintf(stderr, "Out of memory!\n");
        abort();
    }

    // rank is meant to be the rank of this process.
    
    for(i = 0; i < n; ++i)
    {
        send_data[i] = rank;
    }

    // size is meant to be the size of the communicator.

    for(i = 0; i < size; ++i)
    {
        // Send & receive.

		//inds�t noget modulus-bab!
		int destination = (rank+1)%size;
		int source = (rank-1+size)%size;
		printf("rank is %d \n", rank);	
		MPI_Send(&send_data[i], 1, MPI_INT, destination, 123, MPI_COMM_WORLD);
		MPI_Recv(&recv_data[i], 1, MPI_INT, source, 123, MPI_COMM_WORLD, &status);
		printf("rank is %d derp derp derp derp \n", rank);	
        // You'll need that at some point.
        memcpy(send_data, recv_data, n*sizeof(int));
    }
	MPI_Finalize();  
}

\end{lstlisting}

% Your code goes here.
\begin{lstlisting}[basicstyle=\small\sffamily,
keywords={break,case,const,continue,default,else,enum,
for,if,return,switch,while,do,long,void,int,float,double,
char,struct,typedef,include,size\_t},
keywordstyle={\color{blue}},
comment={[l]{//}}, morecomment={[s]{/*}{*/}}, commentstyle=\itshape,
columns={[l]flexible}, numbers=left, numberstyle=\tiny,
frameround=fftt, frame=shadowbox, captionpos=b,
caption={Cycle 2: Blocking, no deadlock.},
label=LST:cycle2]
for(i = 0; i < size; ++i)
    {
        // Send & receive.

		int destination = (rank+1)%size;
		int source = (rank-1+size)%size;
		if (rank%2 == 0){	
		MPI_Send(&send_data[i], 1, MPI_INT, destination, 123, MPI_COMM_WORLD);
		MPI_Recv(&recv_data[i], 1, MPI_INT, source, 123, MPI_COMM_WORLD, &status);
		else{
		MPI_Recv(&recv_data[i], 1, MPI_INT, source, 123, MPI_COMM_WORLD, &status);
		MPI_Send(&send_data[i], 1, MPI_INT, destination, 123, MPI_COMM_WORLD);
		}
        // You'll need that at some point.
        memcpy(send_data, recv_data, n*sizeof(int));
    }
\end{lstlisting}

% Your code goes here.
\begin{lstlisting}[basicstyle=\small\sffamily,
keywords={break,case,const,continue,default,else,enum,
for,if,return,switch,while,do,long,void,int,float,double,
char,struct,typedef,include,size\_t},
keywordstyle={\color{blue}},
comment={[l]{//}}, morecomment={[s]{/*}{*/}}, commentstyle=\itshape,
columns={[l]flexible}, numbers=left, numberstyle=\tiny,
frameround=fftt, frame=shadowbox, captionpos=b,
caption={Cycle 3: Non-blocking, no deadlock.},
label=LST:cycle3]
    for(i = 0; i < size; ++i)
    {
        // Send & receive.

		int destination = (rank+1)%size;
		int source = (rank-1+size)%size;	
		MPI_Request req;
		MPI_Isend(&send_data[i], 1, MPI_INT, destination, 123, MPI_COMM_WORLD, &req);
		MPI_Irecv(&recv_data[i], 1, MPI_INT, source, 123, MPI_COMM_WORLD, &req);
		MPI_Wait(&req, &status);
        // You'll need that at some point.
        memcpy(send_data, recv_data, n*sizeof(int));
    }
\end{lstlisting}



\newpage
\section{Authors}
I/We have solved these exercises independently, and each of us has actively
participated in the development of all of the exercise solutions.
\vspace{1cm}

\noindent
\begin{tabular}{p{70mm}p{70mm}}

%
% Replace ``Name x'' by your name.
%

Dan Duus Th$\emptyset$isen & Rasmus S$\emptyset$gaard Jacobsen \\
\dots\dotfill\dots & \dots\dotfill\dots \\
& \\
& \\
& \\

Mads Vestergaard Carlsen &\\
\dots\dotfill\dots & \dots\dotfill\dots \\
& \\
& \\
& \\
\end{tabular}

\end{document}
