\documentclass{article}
\usepackage{a4wide}
\usepackage{listings,color}
\usepackage[lastexercise]{exercise}
\renewcommand{\ExerciseListHeader}{ 
\textbf{Exercise} \ExerciseHeaderNB:\quad} 
\renewcommand{\QuestionNB}{{\bf Q}:\quad}
\renewcommand{\AnswerListHeader}{ \textbf{A}:\quad} 
\pagestyle{empty}

\title{MVP Assignment 3 - Matrix Inversion}
\date{Due date: 19/4/2010}
\begin{document}
\maketitle


% Leave this commented for answering.
% \newcommand{\question}[1]{#1}
% \newcommand{\answer}[1]{}

% Leave this uncommented for answering.
\newcommand{\question}[1]{}
\newcommand{\answer}[1]{{#1}}

%
% Write your group room and group number here.
%
\answer{
\begin{flushleft}
{\bf Group room:} 2.1.57\\
{\bf Group number:} d204a\\
\end{flushleft}
}

\section{Analysis}

\question{
First we will analyse the code from \texttt{pmatrics.c} that you
discovered in the previous assignment. The example has been updated
for this assignment. The purpose is to parallize it using
\emph{pthreads}. The code supplied contains helpful comments.}

\begin{ExerciseList}
  \Exercise Analysis of \verb+inv_mat(..)+.
  \Question \emph{Identify the different main phases of the matrix
    inversion algorithm.}
  \Answer Initializing the varibales and creates the matrix b as an identity matrix.\\
Performs Gaussian elimination.\\
If the Gaussian elimination is to be performed with pivoting, performs pivoting (this helps with round-off errors).\\
continues with the Gaussian elimination in the Division step by dividing each element of a column (or a row, depending on the matrices orientation) of A, at position (k,j) with an element from A, taken from position (k,k), and likewise for matrix B.\\
	The next step is the Elimination step, at which a row/column, iterated over i, in matrix A (given by A(i,j) is substracted A(i,k)*A(k,j), and likewise with matrix B.\\
\\
Back-substitution is the next step, at which the elements of B are substracted by the following algorithm B(i,j)=B(i,j)-A(i,k)*B(k,j).\\
\\
Finally the result is written into the matrix A.\\
\\
Thus, the following are the main phases of the algorithm:\\
Initialization\\
Pivoting\\
Division\\
Elimination\\
Back-substitution\\
Finalizing\\
\\
Of these, Pivoting, Division, Elimination and Back-substitution can be parallelized.\\
\\
  \Question \emph{What are the data dependencies between the phases?}
  \Answer The data dependencies are pretty clear from the listed sequence of steps. Pivoting needs the data from initialization to be completed, Division needs the data from Pivoting to be completed, and so forth. \\

\end{ExerciseList}

\section{Preliminaries}

\question{
Copy your matrix multiplication code from assignments 1 and 2 in the
relevant matrix multiplication functions place holders.}

\begin{ExerciseList}
\Exercise We will need a barrier.
\Question \emph{Implement it.}
\end{ExerciseList}

\begin{lstlisting}[basicstyle=\small\sffamily,
keywords={break,case,const,continue,default,else,enum,
for,if,return,switch,while,do,long,void,int,float,double,
char,struct,typedef,include,size\_t},
keywordstyle={\color{blue}},
comment={[l]{//}}, morecomment={[s]{/*}{*/}}, commentstyle=\itshape,
columns={[l]flexible}, numbers=left, numberstyle=\tiny,
frameround=fftt, frame=shadowbox, captionpos=b,
caption={Your barrier implementation.},
label=LST:barrier]

/* Implement a barrier here.
   1) Declare your variables. */
   
   pthread_mutex_t barrier_lock;
   pthread_cond_t barrier_condition;
   size_t counter, threadcount;
   
   /* 2) Write the init function to initialize the variables of your barrier. */
   
void init_barrier(size_t arg)
{
counter = 0;
threadcount = arg;
pthread_mutex_init(&barrier_lock, NULL);
pthread_cond_init(&barrier_condition, NULL);
}

/*
   3) Write the barrier function.
*/

void thebarrierfunction()
{
pthread_mutex_lock(&barrier_lock);
counter++;
if(counter == threadcount)
{
	counter = 0;
	pthread_cond_broadcast(&barrier_condition);
}
else
{
	pthread_cond_wait(&barrier_condition, &barrier_lock);
}
pthread_mutex_unlock(&barrier_lock);
}
\end{lstlisting}
% Here goes your barrier: data declarations and relevant functions.

\begin{ExerciseList}
  \Exercise Your threads will be started in
  \texttt{pthread\_inv\_mat}. The function lists what needs to be done
  but it is incomplete.
  \Question \emph{Complete the function.}
\end{ExerciseList}

\begin{lstlisting}[basicstyle=\small\sffamily,
keywords={break,case,const,continue,default,else,enum,
for,if,return,switch,while,do,long,void,int,float,double,
char,struct,typedef,include,size\_t},
keywordstyle={\color{blue}},
comment={[l]{//}}, morecomment={[s]{/*}{*/}}, commentstyle=\itshape,
columns={[l]flexible}, numbers=left, numberstyle=\tiny,
frameround=fftt, frame=shadowbox, captionpos=b,
caption={Function \texttt{pthread\_inv\_mat}.},
label=LST:pinvmat]

invmat_data_t data[nb_threads];     // Data for the threads.
    pthread_t threads[nb_threads];      // The threads.
    size_t index[n];                    // Shared index map.
    sem_t sync1[nb_threads];            // Useful array of semaphores.
    sem_t sync2;                        // Useful semaphore.
    size_t i;                           // Useful var.
    speedup = 1;                        // Inialize the speed-up evaluation.

    // Initialize your semaphores.
	for(i = 0; i < nb_threads; i++)
	{
		sem_init(&sync1[i], 0, nb_threads);
	}
	sem_init(&sync2, 0, nb_threads);
	
    // Initialize your barrier.
	init_barrier(nb_threads);
    // Initialize your data.
	memcpy(a, input, n*n*sizeof(double));
	memset(b, 0, n*n*sizeof(double));
	for(i = 0; i < n; ++i)
	{
	index[i] = i;
	b[i*n+i] = 1.0;
	}
	for(i = 0; i < nb_threads; i++)
    {
		data[i].input = input;
        data[i].a = a;
        data[i].b = b;
        data[i].n = n;
        data[i].id = i;
		data[i].index = index;
    }
	
    // Start your threads.
	for(i = 0; i < nb_threads; i++)
    {
    start_thread(&threads[i],job_inv_mat,&data[i]);
    }
	
    // Join your threads.
	join_threads(threads, nb_threads);
	
	// Result.
    for(i = 0; i < n; ++i)
    {
        for(j = 0; j < n; ++j)
        {
            a[i*n+j] = B(i,j);
        }
    }
	
    // Destroy your semaphores.
	for(i = 0; i < nb_threads; i++)
	{
		sem_destroy(&sync1[i]);
	}
	sem_destroy(&sync2);
\end{lstlisting}
% Here goes your pthread_inv_mat.

\question{
Your thread will run \emph{job\_inv\_mat}. As a preliminatry, copy the
original matrix inversion code here, skipping its local variable
declarations. You will modify it later.}

\section{Design and Implementation}

\question{
You should have identified the different phases of the algorithm in
the previous question before continuing.}

\begin{ExerciseList}

\Exercise 
\answer{Starting and ending.}
\question{We consider thread 0 to be the master thread. We can
parallelize the initializations (copy and reset) or let the master
thread do it. Similarly, we can parallelize the final copy at the end
or not. It does not matter much.}
\Question \emph{Why is it not important to parallelize the
  initialization and the final copy?}
\Answer We only need one instance of these initializations.

\Exercise
\answer{Synchronization from the master.}
\question{The master thread will do the computation that is needed for all
threads to continue.}
\Question \emph{Use a barrier for synchronization.}
\Question $[$Optional$]$ \emph{Implement another synchronization where
  the master thread signals the other thread by a \texttt{sync\_post} on
  the right \texttt{sync1} (array of) semaphores and the other threads
  wait for it.}

\Exercise
\answer{Load balancing.}
\question{The main computation phase is done in parallel. To avoid load
balancing issues, let's use a round-robin distribution of the threads
over the rows (the for-loop on $i$). This can be done simply by
letting thread $t$ execute an iteration if $i\%N==t$, $N$ being the
total number of threads. It is also possible to rearrange the loop to
jump to the right indices directly.}
\Question \emph{What is the load balancing issue that we are
  addressing with the round-robin solution?}
\Answer .. % Your answer here.

\Question \emph{Implement round-robin for the threads.}

\Exercise
\answer{Synchronization to the master.}
\question{
After the main computation phase, the threads need to synchronize
before looping again for the master thread to do its job first. In the
first synchronization, if you did the optional question, the ``other''
threads were waiting for the master but here the master needs to wait
for all the other threads before proceeding.}
\Question \emph{Use a barrier for this second synchronization.}
\Question $[$Optional$]$ \emph{Implement this synchronization with
  semaphores (this time on the sync2 semaphore).}

\Exercise
\answer{Back-substitution.}
\question{The back-substitution can be parallelized too. There is a
data-dependency between every iteration of the $k$ loop (that you
explained in a previous question) so you will use a
barrier.}
\Question \emph{Parallelize the back-substitution, avoiding load
  balancing issues using round-robin like previously on the $i$ loop.}

\Exercise $[$Optional$]$
\answer{timed\_call}
\question{The program outputs different timing results
when it executes. You can read the comments in the
\texttt{timed\_call} function to see what it does.}
\Question $[$Optional$]$ \emph{Explain the different outputs of the
  \texttt{timed\_call} functions.}
\Answer .. % Your answer here.

\end{ExerciseList}

% Your job_inv_mat function goes here.
\begin{lstlisting}[basicstyle=\small\sffamily,
keywords={break,case,const,continue,default,else,enum,
for,if,return,switch,while,do,long,void,int,float,double,
char,struct,typedef,include,size\_t},
keywordstyle={\color{blue}},
comment={[l]{//}}, morecomment={[s]{/*}{*/}}, commentstyle=\itshape,
columns={[l]flexible}, numbers=left, numberstyle=\tiny,
frameround=fftt, frame=shadowbox, captionpos=b,
caption={Matrix inversion with semaphores.},
label=LST:invmat1]

#define Threadcheck if(i % nb_threads == data->id)
    // Typically you need to read your data.

    size_t n = data->n;
    double *a = data->a;
    double *b = data->b;
    size_t *index = data->index;
    int i,j,k;
    double x;

    // And here you go.
	// Gaussian elimination.
    for(k = 0; k < n; ++k)
    {
		if(data->id == 0)
		{
		#if PIVOT
        // Find pivot.
        x = fabs(A(k,k));
        j = k;
        for(i = k+1; i < n; ++i)
        {
            double y = fabs(A(i,k));
            if (y > x)
            {
                j = i;
                x = y;
            }
        }
        // Swap j <-> k.
        i = index[j];
        index[j] = index[k];
        index[k] = i;
		#endif
        // Division step.
        check_divider(x = A(k,k));
        A(k,k) = 1.0;
        for(j = k+1; j < n; ++j) A(k,j) /= x;
        for(j = 0  ; j < n; ++j) B(k,j) /= x;
		}
		thebarrierfunction();
        // Elimination step.
        for(i = k+1; i < n; ++i)Threadcheck
        {
            for(j = k+1; j < n; ++j) A(i,j) -= A(i,k)*A(k,j);
            for(j = 0  ; j < n; ++j) B(i,j) -= A(i,k)*B(k,j);
            A(i,k) = 0.0;
        }
		thebarrierfunction();
    }

    // Back-substitution.
    for(k = n-1; k > 0; --k)
    {
        for(i = k-1; i >= 0; --i)Threadcheck
        {
            for(j = 0; j < n; ++j) B(i,j) -= A(i,k)*B(k,j);
            //A(i,k) = 0.0; -- implicit and not used later.
        }
		thebarrierfunction();
    }
\end{lstlisting}

% Your job_inv_mat function goes here.
\begin{lstlisting}[basicstyle=\small\sffamily,
keywords={break,case,const,continue,default,else,enum,
for,if,return,switch,while,do,long,void,int,float,double,
char,struct,typedef,include,size\_t},
keywordstyle={\color{blue}},
comment={[l]{//}}, morecomment={[s]{/*}{*/}}, commentstyle=\itshape,
columns={[l]flexible}, numbers=left, numberstyle=\tiny,
frameround=fftt, frame=shadowbox, captionpos=b,
caption={Matrix inversion with barriers.},
label=LST:invmat2]
/* Your relevant code goes here. NOT the whole file. */
\end{lstlisting}


\newpage
\section{Authors}
I/We have solved these exercises independently, and each of us has actively
participated in the development of all of the exercise solutions.
\vspace{1cm}

\noindent
\begin{tabular}{p{70mm}p{70mm}}

%
% Replace ``Name x'' by your name.
%

Dan Duus Th$\emptyset$isen & Rasmus S$\emptyset$gaard Jacobsen \\
\dots\dotfill\dots & \dots\dotfill\dots \\
Signature & Signature \\
& \\
& \\

Mads Vestergaard Carlsen & Name 4 \\
\dots\dotfill\dots & \dots\dotfill\dots \\
Signature & Signature \\
& \\
& \\

Name 5 & Name 6 \\
\dots\dotfill\dots & \dots\dotfill\dots \\
Signature & Signature \\
& \\
& \\

Name 7 & Name 8 \\
\dots\dotfill\dots & \dots\dotfill\dots \\
Signature & Signature \\
& \\
& \\
\end{tabular}

\end{document}
